\documentclass[11pt]{beamer}
\definecolor{spoiler}{gray}{0.6}

\usepackage[utf8x]{inputenc}
\usepackage[T2A]{fontenc}
\usepackage[russian]{babel}
\usepackage{xcolor}
\usepackage{MnSymbol}
\usepackage{mathrsfs}
\usepackage{listings}
\usepackage{mathtools}
\usepackage{amssymb,amsmath, dsfont, amsthm}
\usepackage{hyperref}
\hypersetup{unicode=true}
\usepackage{comment}
\usepackage{graphicx}
\usepackage{tikz,makecell}
\usepackage{wrapfig}
\usepackage{xfrac}

\usetheme{Berlin}

\usecolortheme{dolphin}
\title[Лекция <<Вокруг $\sqrt{2}$>]
	{\bfseries Лекция <<Вокруг $\sqrt{2}$>>}
\author[Тодоров Е.И.]{Тодоров Е.И.}
\institute{Фонд <<Время науки>>}
\date{\today}

\newtheorem{Th}{Теорема}
\newtheorem{Def}{Определение}
\newtheorem{Ex}{Пример}
\newtheorem{Proposition}{Предложения}
\newtheorem{Property}{Свойство}
\newtheorem{Prof}{Доказательство}
\newtheorem{Profcont}{Доказательство (продолжение)}

\newcommand\fram[2]{\begin{frame}{\bf #1} #2 \end{frame}}
\newcommand{\comm}[1]{$\backslash$\texttt{#1}}
\def\scolon{\rlap{,}\raisebox{0.8ex}{,} }
\def\mitem{\medskip\item}
\newcommand{\myref}[2]{\href{#1}{\texttt{\underline{#2}}}}
\newcommand{\set}[1]{\left\{ #1 \right\}}


\begin{document}
\section{ }
\begin{frame}\titlepage\end{frame}

\fram{Эта презентация онлайн}{
\begin{center}
Зачем фотографировать презентацию, \\
		когда её можно скачать? \\ [0.4cm]
		\includegraphics[width=5cm]{chainfrac-link}
\end{center}
}

\section{A4}
\fram{Лист A4}{
\begin{wrapfigure}{L}{0.5\textwidth}
\vspace{-17mm}
  \begin{center}
    \tikz[scale=0.022]{
    \draw[black,thick] (0,0)--(210,0)--(210,297)--(0,297)--(0,0);
    \draw[black,thick,dashed] (0,148.5)--(210,148.5);
    \draw (105,297) node[anchor=north] {$a$};
    \draw (210,148.5) node[anchor=west] {$b$};
    \draw (0,74.25) node[anchor=west] {$\frac{b}{2}$};
    \draw (0,222.75) node[anchor=west] {$\frac{b}{2}$};
    }
  \end{center}
\end{wrapfigure}

Свойство сторон листа A4 \linebreak (всех листов формата A):
$$a:b = \frac{b}{2}:a,$$

откуда
$$b^2 = 2a^2.$$
}

\section{$\sqrt{2}$}
\fram{}{
\
\vspace{-15mm}
\begin{Def}
Для числа $s \ge 0$ квадратным корнем из $s$ будем называть такое число $t \ge 0$, что $t^2 = s$. Пишут $t = \sqrt{s}$.
\end{Def}
\begin{Ex}
\vspace{-7mm}
$$\sqrt{1} = 1, \qquad \sqrt{4} = 2, \qquad \sqrt{\frac{25}{4}} = \frac{5}{2}, \qquad \sqrt{2} = 1.414213\ldots$$
\vspace{-5mm}
\end{Ex}
\begin{Property}
\vspace{-4mm}
$$s \ge 0 \qquad \left(\sqrt{s}\right)^2 = s = \sqrt{s^2}$$
\vspace{-6mm}
\end{Property}
\begin{Property}
\vspace{-4mm}
$$s, r \ge 0 \qquad \sqrt{s \cdot r} = \sqrt{s} \cdot \sqrt{r}$$
\vspace{-6mm}
\end{Property}
}

\fram{$\sqrt{2}$}{
Применим описанные свойства к соотношению $b^2 = 2a^2$:
\begin{align*}
    \sqrt{b^2} &= \sqrt{2a^2}\\
    \sqrt{b^2} &= \sqrt{2}\sqrt{a^2}\\
    b &= \sqrt{2}a\\
    \frac{b}{a} &= \sqrt{2}.
\end{align*}
\textbf{Вопрос:} можно ли найти такие целые $a$ и $b$?
}

\section{Числовые множества}
\fram{Числовые множества}{
\begin{itemize}
    \item Натуральные числа: $\mathbb{N} = \set{1, 2, 3, 4, 5, \ldots}$ \vspace{3mm}
    \item Целые числа: $\mathbb{Z} = \set{\ldots, -2, -1, 0, 1, 2, \ldots}$ \vspace{3mm}
    \item Попробуем теперь собрать все дроби $\frac{b}{a}$: \vspace{2mm}
    \begin{itemize}
        \item $\frac{-3}{5} = \frac{3}{-5}$. Берём $b$ любое, а $a$ --- только положительное. \vspace{2mm}
        \item $\frac{1}{2} = \frac{2}{4} = \frac{3}{6} = \frac{4}{8} = \ldots$.
        Берём только несократимые дроби. То есть у $b$ и $a$ не должно быть общих делителей. \vspace{1mm}
    \end{itemize}
    \item Рациональные числа (дроби): 
    $$\mathbb{Q} = \set{\frac{b}{a} \,\,\middle|\,\, b - \text{целое}, a - \text{натуральное}, \frac{a}{b} - \text{несократимая}}.$$
\end{itemize}
}

\section{Теорема Аристотеля}
\fram{Теорема Аристотеля}{
\begin{Th}
Число $\sqrt{2}$ не живёт в $\mathbb{Q}$. То есть не существует таких целых $a$ и $b$, что $\frac{b}{a} = \sqrt{2}$, где $\frac{b}{a}$ --- \underline{несократимая дробь}.
\end{Th}
\begin{Prof}
Пусть такие $a$ и $b$ найдутся. Тогда \vspace{-4mm}

$$\frac{b}{a} = \sqrt{2},$$ \vspace{-6mm}

что эквивалентно \vspace{-6mm}

$$b^2 = 2a^2$$ \vspace{-9mm}

\begin{itemize}
\item $b^2$ должно делится на 2;
\item \underline{$b$ делится на 2};
\item найдётся такое $k$, что $b = 2k$;
\end{itemize}
\end{Prof}
}

\fram{}{
\begin{Profcont}
\begin{itemize}
\item $b^2 = 4k^2 = 2a^2$;
\item $a^2 = 2k^2$;
\item $\ldots$;
\item \underline{$a$ делится на 2};
\item противоречие.
\end{itemize}
\end{Profcont}

Значит, не существует таких целых $a$ и $b$, что $\frac{b}{a} = \sqrt{2}$.

Значит, уравенеие $b^2 = 2a^2$ не имеет решений для целых $a$ и $b$.
}


\fram{Иллюстрация рациональности}{
\ \vspace{-11mm}
\begin{center}
    \tikz[scale=0.43]{
    \foreach \x in {1,2,3,4,5,6,7,8,9,10,11,12,13,14,15,16,17,18,19,20,21,22,23,24}
    {
        \foreach \y in {1,2,3,4,5,6,7,8,9,10,11,12}
        {
        \draw (\x, \y) circle (2pt);
        }
        \draw (\x,-0.1)--(\x,0.1) node[anchor=north] {\tiny \x};
    }
    \foreach \y in {1,2,3,4,5,6,7,8,9,10,11,12}
    {
        \draw (-0.1,\y)--(0.1,\y) node[anchor=east] {\tiny \y};
    }
    \draw[black, ->] (0,0)--(25,0);
    \draw[black, ->] (0,0)--(0,13);
    \draw (25,0) node[anchor=north] {$b$};
    \draw (0,13) node[anchor=east] {$a$};
    \draw[black] (0,0)--(13,13);
    \foreach \x in {1,2,3,4,5,6,7,8,9,10,11,12}{
    \filldraw (\x,\x) circle (2.5pt);}
    \draw (13,13) node[anchor=south] {$\frac{b}{a} = 1$};
    \draw[blue] (0,0)--(6.5,13);
    \foreach \x in {1,2,3,4,5,6}{
    \filldraw[blue] (\x,2*\x) circle (2.5pt);}
    \draw[blue] (6.5,13) node[anchor=south] {$\frac{b}{a} = \frac{1}{2}$};
    \draw[green] (0,0)--(22.45454545,13);
    \filldraw[green] (19,11) circle (2.5pt);
    \draw[green] (22.45454545,13) node[anchor=south] {$\frac{b}{a} = \frac{19}{11}$};
    \draw[red] (0,0)--(18.3847763109,13);
    \draw[red] (18.3847763109,13) node[anchor=south] {$\frac{b}{a} = \sqrt{2}$};
%    \draw (0,74.25) node[anchor=west] {$\frac{b}{2}$};
%    \draw (0,222.75) node[anchor=west] {$\frac{b}{2}$};
    }
  \end{center}
}

\section{Задачи}
\fram{Задача 1}{
\Large
Может ли иррациональное число в иррациональной степени быть рациональным?
}

\fram{Решение 1}{
Да, может.

Рассмотрим число 

$$\left(\sqrt{2}\right)^{\sqrt{2}}.$$

Если оно рационально, то мы победили.

Если нет, то возведём и его в степень $\sqrt{2}$:

$$\left(\left(\sqrt{2}\right)^{\sqrt{2}}\right)^{\sqrt{2}} = \left(\sqrt{2}\right)^{\sqrt{2}\cdot\sqrt{2}} = \left(\sqrt{2}\right)^2 = 2.$$

И теперь мы точно победили.
}

\fram{Задача 2}{
\Large
Генерал построил своё подразделения в форме квадрата? Может ли он перестроить это же подразделение в виде двух квадратов?
}

\section{Цепная дробь}
\fram{Цепная дробь}{
Рассмотрим дробь
$$1+\frac{1}{2+\frac{1}{2+\frac{1}{2+\frac{1}{2+\ldots}}}} = x \qquad \Longleftrightarrow \qquad 2+\frac{1}{2+\frac{1}{2+\frac{1}{2+\frac{1}{2+\ldots}}}} = 1+x.$$

Заменим знаменатель первой дроби на $x$:\vspace{-5mm}

\begin{align*}
    2 + \frac{1}{x+1} &= 1+x\\
    \frac{1}{x+1} &= x-1\\
    1 &= (x-1)(x+1)\\
    1 &= x^2 - 1\\
    x^2 &= 2\\
    x &= \sqrt{2}.
\end{align*}
}

\section{Приближенные значения}
\fram{Приближенные значения $\sqrt{2}$}{
Обрубая дробь с предыдущего слайда на том или ином этаже мы получим последовательность \textit{приближенных значений} $\sqrt{2}$:

$$\frac{1}{1}, \quad \frac{3}{2}, \quad \frac{7}{5}, \quad \frac{17}{12}, \quad \frac{41}{29}, \ldots$$

\begin{itemize}
    \item Каждая следующая такая дробь приближает $\sqrt{2}$ всё лучше и лучше.
    \item Если $\frac{p}{q}$ и $\frac{P}{Q}$ --- две последовательные дроби из ряда выше, то
    $$P = p+2q, \qquad\qquad Q = p+q.$$
\end{itemize}
}

\fram{}{
\vspace{-10mm}
\begin{Property}
Для каждой дроби $\frac{P}{Q}$ из ряда выше выполнено: \vspace{-4mm}

$$P^2 - 2Q^2 = \pm 1,$$ \vspace{-8mm}

что экививалентно  \vspace{-6mm}

$$P^2 = 2Q^2 \pm 1.$$ \vspace{-6mm}


\end{Property}

\begin{Prof}
Пусть для дроби $\frac{p}{q}$ это выполнено, то есть $p^2 - 2q^2 = \pm 1$. 

Рассмотрим следующую за $\frac{p}{q}$ дробь $\frac{P}{Q}$. Для неё верны соотношения $P = p+2q$ и $Q = p+q$. Тогда \vspace{-7mm}

\begin{align*}
    P^2 - 2Q^2 = (p+2q)^2 - 2(p+q)^2 =\\
    p^2 + 4pq+4q^2 - 2p^2 - 4pq - 2q^2 =\\
    2q^2 - p^2 = -(p^2-2q^2) = \mp 1.
\end{align*} \vspace{-7mm}
\end{Prof}
}

\fram{Опять задача}{
Вернёмся к задаче о генерале и подразделениях. Изначальная задача имеет отрицательный ответ, т.к. для любых целых $a$ и $b$ $b^2 \neq 2a^2$.

Но если взять $b = p$ и $a = q$ из дробей выше, мы гарантированно можем добиться разницы между $b^2$ и $2a^2$ всего в одного солдата.\vspace{-4mm}

\begin{center} \tikz[scale=0.15]
{
   \foreach \x in {1,2,3,4,5,6,7,8,9,10,11,12,13,14,15,16,17}
   {
    \foreach \y in {1,2,3,4,5,6,7,8}
   {
    \filldraw[fill opacity=0.6, red] (\x, \y) rectangle (\x cm + 1 cm, \y cm + 1 cm);
   }
   }
   \foreach \x in {1,2,3,4,5,6,7,8}
   {
    \filldraw[fill opacity=0.6, red] (\x, 9) rectangle (\x cm + 1 cm, 9 cm + 1 cm);
   }
   \foreach \x in {10,11,12,13,14,15,16,17}
   {
    \filldraw[fill opacity=0.6, blue] (\x, 9) rectangle (\x cm + 1 cm, 9 cm + 1 cm);
   }
   \foreach \x in {1,2,3,4,5,6,7,8,9,10,11,12,13,14,15,16,17}
   {
    \foreach \y in {10,11,12,13,14,15,16,17}
   {
    \filldraw[fill opacity=0.6, blue] (\x, \y) rectangle (\x cm + 1 cm, \y cm + 1 cm);
   }
   }
   \draw (20,9.3) node {\huge $\rightarrow$};
   \foreach \x in {22,23,24,25,26,27,28,29,30,31,32,33}
   {
    \foreach \y in {-4,-3,-2,-1,0,1,2,3,4,5,6,7}
   {
    \filldraw[fill opacity=0.6, red] (\x, \y) rectangle (\x cm + 1 cm, \y cm + 1 cm);
   }
   \foreach \y in {11,12,13,14,15,16,17,18,19,20,21,22}
   {
    \filldraw[fill opacity=0.6, blue] (\x, \y) rectangle (\x cm + 1 cm, \y cm + 1 cm);
   }
   }
   \draw[fill opacity=0.6] (23, 9) rectangle (23 cm + 1 cm, 9 cm + 1 cm);
}
\end{center}

}

\section{ }
\fram{}{
\begin{center}
    {\Large Спасибо за внимание!}
    \vspace{0.5cm} \hrule \vspace{0.3cm}

Что мы узнали сегодня:
\begin{enumerate}
	\item откуда взялось число $\sqrt{2}$ и почему оно иррационально;
	\item что такое Теорема Аристотеля;
	\item как цепные дроби приближают числа;
	\item как почти разбить один квадрат на два равных.
\end{enumerate} \vspace{0.3cm}

    \hrule\vspace{0.5cm}
    Задать вопрос автору:
    \myref{mailto:todzhe@mail.ru}{todzhe@mail.ru}
\end{center}
}
\end{document}