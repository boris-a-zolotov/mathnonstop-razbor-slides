\begin{frame} \frametitle{Концепция заданий олимпиады}
\begin{itemize}
	\item Невозможно успеть решить все задачи, \medskip
	\item Есть задачи как для совсем неопытных участников, \\ так и для занимавшихся в кружках. \medskip
	\item Каждая задача разделена на пункты A, B и C, берётся \\ максимум баллов. \medskip
	\item Профильные варианты: новая обстановка для учеников матшкол.
\end{itemize}
\end{frame}

\begin{frame} \frametitle{Простые задания}
\usl{2020-4-4B}{
	У Вани есть доски для паркета размером $20 \times 10$ сантиметров, их можно распиливать пополам. Как Ване покрыть этими досками пол квадратной комнаты $1\text{ метр}\times 1\text{ метр}$ так, чтобы не было швов длиной более 30 сантиметров ни в одном из направлений?
}

\begin{center} \vspace{-0.4cm} \tikz[scale=0.4]
{
   \foreach \x / \y in {
	1/2, 1/6, 1/10, 2/1, 2/5, 2/9, 3/4, 3/8, 4/3, 4/7, 5/2, 5/6, 5/10, 6/1, 6/5, 6/9, 7/4, 7/8,
	8/3, 8/7, 9/1, 9/2, 9/6, 9/10
   } 
   \draw[fill opacity=0.6] (\x, \y) rectangle (\x cm + 2 cm, \y cm + 1 cm);
   \foreach \x / \y in {
	1/4, 1/8, 2/3, 2/7, 3/2, 3/6, 4/1, 4/5, 4/9, 5/4, 5/8, 6/3, 6/7, 7/2, 7/6, 8/1, 8/5, 8/9, 9/4, 9/8, 10/3, 10/7
   } 
   \draw[fill opacity=0.6] (\x, \y) rectangle (\x cm + 1 cm, \y cm + 2 cm);
   \foreach \x / \y in {
	1/1, 1/3, 1/7, 3/10, 5/1, 7/10, 10/5, 10/9
   } 
   \draw[fill opacity=0.6] (\x, \y) rectangle (\x cm + 1 cm, \y cm + 1 cm);
   \draw[thick,fill opacity=0.6] (1, 1) rectangle (11 cm, 11 cm);
}  \end{center}
\end{frame}

\begin{frame} \frametitle{Сложные задания}
Выпишем все числа от одного до десяти — но не в привычном порядке возрастания, а в алфавитном порядке: восемь, два, девять, десять, один, пять, семь, три, четыре, шесть. \medskip

\usl{2020-6-4B}{
Числа от 1 до 10'000'000'000 (десять миллиардов) выписали в алфавитном порядке. Перечислите первые десять из них.
}

(1) 18 (2) 18 миллионов (3) 18 миллионов 18 (4) 18 миллионов 18 тысяч \\
(5) 18 миллионов 18 тысяч 18 (6) \ldots восемь (7) \ldots восемьдесят \\
(8) \ldots 88 (9) \ldots 82 (10) \ldots 89.
\end{frame}

\begin{frame} \frametitle{Профильные задания}
{\it Система високосных лет} для числа $t$ — это последовательность натуральных чисел $(a_0, a_2, a_3, \ldots, a_n)$ такая, что $a_{i+1}$ делится на $a_i$, а также
	$$\frac{1}{a_0} - \frac{1}{a_1} + \frac{1}{a_2} - \frac{1}{a_3} + \ldots
	     + (-1)^{n} \cdot \frac{1}{a_n}\ =\ t.$$ \smallskip

Какой могла бы быть система високосных лет, если бы длина года составляла 365.21875, 365.17, 365.33 дней? Для любого ли рационального числа существует система високосных лет?
\end{frame}
