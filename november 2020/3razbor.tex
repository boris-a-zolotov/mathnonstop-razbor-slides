\documentclass[aspectratio=1610,12pt,notheorems]{beamer}

\usepackage[utf8x]{inputenc} \usepackage[russian]{babel}
\usepackage{amsmath,amssymb,amsthm,mathtools}
\usepackage{graphicx,caption,subcaption}
\usepackage{hyperref,natbib}
\usepackage{tikz,xcolor,colortbl,makecell}
\usepackage{algorithm,algpseudocode}
\usepackage{qrcode}

\usetikzlibrary{arrows,backgrounds,patterns,%
	matrix,shapes,fit,calc,shadows,plotmarks,snakes}

\theoremstyle{plain}
\newtheorem{theorem}{Теорема}
\newtheorem{lemma}[theorem]{Лемма}

\theoremstyle{definition}
\newtheorem{definition}{Определение}
\newtheorem{problem}{Задача}

\usetheme[height=0.97cm]{Rochester}
\usecolortheme{dolphin}

\definecolor{hard}{RGB}{145,55,55}
\definecolor{mnsgold}{RGB}{240,235,220}

\setbeamercolor{headline}{bg=hard,fg=mnsgold}
\setbeamercolor*{frametitle}{parent=headline}

\setbeamercolor{structure}{fg=hard}
\setbeamercolor{subsection in head/foot}{bg=white,fg=hard}
\setbeamercolor{section in head/foot}{bg=hard,fg=mnsgold}
\setbeamercolor{block title}{bg=hard,fg=mnsgold}

\setbeamertemplate{navigation symbols}{}

\parskip=2.6mm
\def\ll{\left(} \def\rr{\right)}
\def\lag{\left\langle} \def\rag{\right\rangle}

\definecolor{failpos}{RGB}{230,30,20}
\definecolor{initpos}{RGB}{30,20,220}
\definecolor{turna}{RGB}{100,240,110}
\definecolor{turnb}{RGB}{250,140,110}

\newcommand{\vseper}{\vphantom{$\int_{0_0}^{0^0}$}}

\newcommand{\singlepayoff}[2]{\tikz{
	\draw (0,0) node{\vseper #1}; \draw (0.8,0.8) node{\vseper #2};
}}

\newcommand{\rowdescription}[1]{\tikz{
	\draw (0,0) node{\vseper #1}; \draw (0,0.8) node{\vseper};
}}

\newcommand{\coldescription}[1]{\tikz{
	\draw (0,0) node{\vseper}; \draw (0.8,0) node{\vseper #1};
}}

\newcommand{\myref}[2]{\href{#1}{\texttt{\underline{#2}}}}

\def\mitem{\medskip\item}
\def\ps{\\ [0.65cm]} \linespread{1.16}
\def\fram#1#2{\begin{frame}\frametitle{#1}#2\end{frame}}
\def\usl#1#2{\begin{block}{#1} #2 \end{block} \medskip\pause}
\def\uslnp#1#2{\begin{block}{#1} #2 \end{block} \medskip}
\def\mov#1#2{\begin{scope}[xshift = #1 cm] #2 \end{scope};}

\def\divsby{\mathrel{\rlap{.}\rlap{\raisebox{0.55ex}{.}}\raisebox{1.1ex}{.}}}
\definecolor{starvert}{RGB}{110,175,230}
\newcommand{\vfi}{\varphi}
\newcommand{\litem}{\vspace{0.5cm}\item}

\newcommand{\nkstar}[2]{
	\foreach \i in {1,...,#1} {\draw[thick] (90 + 360 * \i / #1 : 3.5)
		-- (90 + 360 * \i / #1 + 360 * #2 / #1 : 3.5);}
	\foreach \i in {1,...,#1} {\fill[starvert] (90 + 360 * \i / #1 : 3.5) circle[radius=0.16cm];}
}

\title[Solutions of MNS]
    {\bfseries Решения избранных задач}

\author[\ ]
	{Б. А. Золотов,\quad «Математика НОН-СТОП»\\ \vspace{0.3cm}
		{\small Фонд «Время Науки»}}

\institute[\ ]{\ }

\date{4 декабря 2020}

%%%%%%%%%%%%%%
%%%%%%%%%%%%%%

\begin{document}

\frame{\titlepage}


%%%%%% 5 класс, разрезания
\begin{frame} \frametitle{Разрезания}
	\usl{2020-5-1C}{
		Можно ли нарисовать на клетчатом листе бумаги такую фигуру, которую можно разрезать по линиям сетки на две {\itshape одинаковые} фигуры двумя способами~— причём фигуры в первом и во втором способе были бы одни и те же, но линии разреза выглядели бы по-разному?
	}

\begin{center} \tikz[scale=0.75]
{
   \foreach \x / \y in {
	1/3, 2/3, 3/3, 3/4, 6/1, 6/2, 6/3, 7/3
   } 
	\draw[thick,fill opacity=0.6] (\x, \y) rectangle (\x cm + 1 cm, \y cm + 1 cm);
   {
   \foreach \x / \y in {
	1/1, 1/2, 2/2, 3/2, 7/2, 8/2, 8/3, 8/4
   } 
	\draw[thick,fill=gray,fill opacity=0.6] (\x, \y) rectangle (\x cm + 1 cm, \y cm + 1 cm);
	\node at (1.5,2.5) {$A$};
	\node at (3.5,3.5) {$B$};
	\node at (6.5,3.5) {$B$};
	\node at (8.5,2.5) {$A$};
}
} 
\end{center}
\end{frame}


%%%%%% 5 класс, семнашка, 4 пары
\begin{frame} \frametitle{Семнадцатый независимый}

\usl{2020-5-3A}{
	Песню каждого участника оценивает 15 судей. Судья ставит каждому участнику в паре от 0 до 22 баллов и отдаёт свой голос участнику, которому поставил больше баллов. В паре объявляется победителем тот участник, которому отдано больше голосов. Может ли быть так, что победитель в паре набрал меньше баллов, чем проигравший, несмотря на перевес в голосах?
}

\end{frame}

\begin{frame} \frametitle{Семнадцатый независимый}

\newcommand{\vs}{&\hspace{-3mm}:&\hspace{-3mm}}
\begin{center} \begin{tabular}{|l|rcl|rcl|c|c|}
	\hline
	Участник & \multicolumn{3}{c|}{Победы}
		& \multicolumn{3}{c|}{Проигрыши}
		& Баллы & Голоса \\ \hline \hline
	Победитель 1 & 1 \vs 0 & 0 \vs 22 & 11 & 11 \\ \hline
	Победитель 2 & 2 \vs 0 & 0 \vs 16 & 20 & 10 \\ \hline
	Победитель 3 & 3 \vs 0 & 0 \vs 12 & 27 & 9 \\ \hline
	Победитель 4 & 4 \vs 0 & 0 \vs 9 & 32 & 8 \\ \hline \hline
	Проигравший 4 & 9 \vs 0 & 0 \vs 4 & 63 & 7 \\ \hline
	Проигравший 3 & 12 \vs 0 & 0 \vs 3 & 72 & 6 \\ \hline
	Проигравший 2 & 16 \vs 0 & 0 \vs 2 & 80 & 5 \\ \hline
	Проигравший 1 & 22 \vs 0 & 0 \vs 1 & 88 & 4 \\ \hline
\end{tabular} \end{center}

\end{frame}


%%%%%% 6 класс, сортировка
\begin{frame} \frametitle{Сортировка}
Выпишем все числа от одного до десяти — но не в привычном порядке возрастания, а в алфавитном порядке: восемь, два, девять, десять, один, пять, семь, три, четыре, шесть. \medskip

\usl{2020-6-4B}{
Числа от 1 до 10'000'000'000 (десять миллиардов) выписали в алфавитном порядке. Перечислите первые десять из них.
}

(1) 18 (2) 18 миллионов (3) 18 миллионов 18 (4) 18 миллионов 18 тысяч \\
(5) 18 миллионов 18 тысяч 18 (6) \ldots восемь (7) \ldots восемьдесят \\
(8) \ldots 88 (9) \ldots 82 (10) \ldots 89.
\end{frame}


%%%%%% Профиль
\begin{frame} \frametitle{Профильные задания}
{\it Система високосных лет} для числа $t$ — это последовательность натуральных чисел $(a_0, a_2, a_3, \ldots, a_n)$ такая, что $a_{i+1}$ делится на $a_i$, а также
	$$\frac{1}{a_0} - \frac{1}{a_1} + \frac{1}{a_2} - \frac{1}{a_3} + \ldots
	     + (-1)^{n} \cdot \frac{1}{a_n}\ =\ t.$$ \smallskip

Какой могла бы быть система високосных лет, если бы длина года составляла 365.21875, 365.17, 365.33 дней? Для любого ли рационального числа существует система високосных лет?
\end{frame}

\end{document}
