\section[Самые простые задачи]{5. Problems For Peasants}

\begin{frame}
\frametitle{Незакрученный удар}

\usl{2019-4-2A}{Шарик катается
по прямоугольнику, не замедляясь. Когда он подъезжает к краю
прямоугольника, он отскакивает от него и продолжает движение.
В каком положении окажется шарик, будучи запущенным
из центра прямоугольника на рисунке, после того как он проедет
24 клетки по диагонали?}

\def\ballandthick{
	\filldraw (0,0) circle[radius=1.2mm];
	\draw[very thick,->] (0,0) -- (-0.72,0.72);
	\draw[very thick] (-1,-1.5) rectangle (1,1.5);
}

\begin{center} \tikz[scale=1.04]{
	\draw (-1.2,0) node[left]{\small \begin{minipage}{4.2cm}\ \end{minipage}};
	\foreach \x in {-2,...,2} {\draw[color=gray] (-0.5 * \x cm, -1.5) -- (-0.5 * \x cm, 1.5);}
	\foreach \x in {-3,...,3} {\draw[color=gray] (-1, -0.5 * \x cm) -- (1, -0.5 * \x cm);}
	\draw[color=gray, thick, dotted] (0,0) -- (-1,1) -- (-0.5,1.5) -- (0.3,0.7);
	\ballandthick;
\pause
	\draw[color=gray, thick, dotted] (0.3,0.7) -- (1,0) -- (-0.5,-1.5) -- (-1,-1) -- (1,1)
	    -- (0.5,1.5) -- (-1,0) -- (0.5,-1.5) -- (1,-1) -- (0,0);
	\ballandthick; \draw[thick] (-1,0) -- (1,0);
	\draw (1.35,0) node[right]{\small \begin{minipage}{5.6cm} Раз в 6 клеток пересекает \\
	    горизонтальную среднюю линию \end{minipage}};
} \end{center} \end{frame}

\begin{frame} \frametitle{Кирпичей требуют наши сердца}

\usl{2020-4-4B}{
	У Вани есть доски для паркета размером $20 \times 10$ сантиметров, их можно распиливать пополам. Как Ване покрыть этими досками пол квадратной комнаты $1\text{ метр}\times 1\text{ метр}$ так, чтобы не было швов длиной более 30 сантиметров ни в одном из направлений?
}

\begin{center} \vspace{-0.4cm} \tikz[scale=0.4]
{
   \foreach \x / \y in {
	1/2, 1/6, 1/10, 2/1, 2/5, 2/9, 3/4, 3/8, 4/3, 4/7, 5/2, 5/6, 5/10, 6/1, 6/5, 6/9, 7/4, 7/8,
	8/3, 8/7, 9/1, 9/2, 9/6, 9/10
   } 
   \draw[fill opacity=0.6] (\x, \y) rectangle (\x cm + 2 cm, \y cm + 1 cm);
   \foreach \x / \y in {
	1/4, 1/8, 2/3, 2/7, 3/2, 3/6, 4/1, 4/5, 4/9, 5/4, 5/8, 6/3, 6/7, 7/2, 7/6, 8/1, 8/5, 8/9, 9/4, 9/8, 10/3, 10/7
   } 
   \draw[fill opacity=0.6] (\x, \y) rectangle (\x cm + 1 cm, \y cm + 2 cm);
   \foreach \x / \y in {
	1/1, 1/3, 1/7, 3/10, 5/1, 7/10, 10/5, 10/9
   } 
   \draw[fill opacity=0.6] (\x, \y) rectangle (\x cm + 1 cm, \y cm + 1 cm);
   \draw[thick,fill opacity=0.6] (1, 1) rectangle (11 cm, 11 cm);
} 
\end{center}

\end{frame}

\begin{frame} \frametitle{Интереснее, чем кажутся}

\usl{2020-6-6A}{
	Прислонив к зеркалу правый край экрана калькулятора, Серёжа заметил, что отражение некоторых чисел в зеркале тоже можно прочитать как (возможно, другое) число. Например, из 281 получается 185. Можно ли прочитать отражение чисел:
     \[ 180;\quad 205;\quad 12851;\quad 369;\quad 31813\text{?} \]
} \vspace{-0.4cm}

\[ 1 \to 1,\ 2 \to 5,\ 3 \to ?,\ 4 \to ?,\ 5 \to 2,\ 6 \to ?,\ 7 \to ?,\ 8 \to 8,\ 9 \to ?,\ 0 \to 0. \]

\end{frame}
