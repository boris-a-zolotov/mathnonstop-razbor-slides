\section[Самые сложные задачи]{4. Difficult Problems}

\begin{frame} \frametitle{Сортировка}

Выпишем все числа от одного до десяти — но не в привычном порядке возрастания, а в алфавитном порядке: восемь, два, девять, десять, один, пять, семь, три, четыре, шесть. \medskip

\usl{2020-6-4B}{
Числа от 1 до 10'000'000'000 (десять миллиардов) выписали в алфавитном порядке. Перечислите первые десять из них.
}

(1) 18 (2) 18 миллионов (3) 18 миллионов 18 (4) 18 миллионов 18 тысяч \\
(5) 18 миллионов 18 тысяч 18 (6) \ldots восемь (7) \ldots восемьдесят \\
(8) \ldots 88 (9) \ldots 82 (10) \ldots 89.

\end{frame}

\begin{frame} \frametitle{Бинарные операции}
\usl{2021-7-9A}{
Определим операцию $\star$ для положительных чисел $a$ и $b$ следующим образом: $a\star b=\frac{ab}{a+b}$.\\
Докажите, что $(a\star b)\star c=a\star (b\star c)$.
}
\usl{2021-7-9B}{
Посчитайте $1\star\left(2\star\left(4\star\left(\ldots\star\left(256\star(512\star1024)\right)\ldots\right)\right)\right)$.
}
Перерасставим скобки: \vspace{-3mm}
	\[ (\ldots((1\star2)\star4)\star\ldots)\star1024 = \left(\ldots\left(\frac{2}{3}\star4\right)\star\ldots\right)\star1024 =\] \vspace{-3mm}
	\[ \left(\ldots\frac{4}{7}\star\ldots\right)\star1024 = \frac{1024}{2047}.\]
\end{frame}

\begin{frame} \frametitle{То ли индийцам, то ли арабам}

\usl{2021-7-3B}{Любое число сравнимо по модулю 7 с нулём или $10^k$}

\usl{2021-7-3C}{Придумать число, которое при умножении на 2 продолжает состоять из тех же цифр, но в другом порядке}

\begin{enumerate}
	\item Период чисел $\frac27 \ldots \frac67$ — сдвиг периода числа $\frac17$.
	\mitem Нет переноса через разряд при умножении, потому что тогда изменилась бы целая часть.
\end{enumerate}

\end{frame}
