\begin{frame} \frametitle{Формула для функции Эйлера}
	$n = p_1^{k_1} \cdot p_2^{k_2} \cdot \ldots \cdot p_m^{k_m}$ — разложение на простые.\\ [0.8cm]

   \begin{theorem}
	\[ \vfi(n) = n \cdot \frac{p_1 - 1}{p_1} \cdot \frac{p_2 - 1}{p_2} \cdot \ldots \cdot \frac{p_m - 1}{p_m} \]\ 
   \end{theorem}\vspace{3mm} \pause

	\textcolor{hard}{\bf Идея доказательства: } выкинуть все числа, делящиеся на $p_1$.	Затем\\
	выкинуть все числа, делящиеся на $p_2$, которые не были выкинуты ранее\\
	 (например, не выкинуть $p_1 p_2$ ещё раз). Затем выкинуть все числа,\\
	 делящиеся на $p_3$...
\end{frame}

\begin{frame} \frametitle{Выкидываем делящиеся на $p_1$}
	Каждое $p_1$-ое число делится на $p_1$. Разделим $n$ на блоки\\
	по $p_1$ чисел, из каждого выкинем по одному числу.\\
	Останется \vspace{-5mm}
	
	\[ n \cdot \frac{p_1 - 1}{p_1}. \] \vspace{3mm}
	
	Это должно напоминать формулу $\vfi\ll p^k \rr$.
\end{frame}

\begin{frame} \frametitle{Выкидываем делящиеся на $p_k$}
	Требуется оставить числа, которые не делятся на $p_k$,\\
	а также не делятся ни на одно из чисел $p_1,\ldots,p_{k-1}$.\medskip
	
	$x \divsby p_i\ \Longleftrightarrow\ p_k \cdot x \divsby p_i$.\quad Отсюда доля чисел, делящихся\\
	на $p_1,\ldots,p_{k-1}$, одинакова {\it вообще} и среди чисел, кратных $p_k$.\\
	А, значит, и среди чисел, не кратных $p_k$.
	
	\[ n \cdot
		\underbrace{ \frac{p_k-1}{p_k} }_{\begin{minipage}{2.1cm}\centering\scriptsize
			считаем \\ долю среди \\ не кратных $p_k$
		\end{minipage}}
		\cdot \ll
		\underbrace{ \frac{p_1 - 1}{p_1} \cdot \ldots \cdot
		     \frac{p_{k-1} - 1}{p_{k-1}}}_{\begin{minipage}{3.5cm}\centering\scriptsize
			выяснили ранее
		     \end{minipage}}
		\rr\]
\end{frame}

\begin{frame} \frametitle{Мультипликативность функции Эйлера}
\[ \vfi(n) = n \cdot \frac{p_1 - 1}{p_1} \cdot \frac{p_2 - 1}{p_2} \cdot \ldots \cdot \frac{p_m - 1}{p_m} \] \vspace{5mm}

\begin{theorem} \vspace{2mm}
	Если числа $a$ и $b$ взаимно просты, то $\vfi(a \cdot b) = \vfi(a) \cdot \vfi(b)$.  \vspace{3mm}
\end{theorem} \vspace{5mm}

\textcolor{hard}{\bf Упражнение: } сперва доказать мультипликативность $\vfi$,\\ а потом из неё вывести формулу.
\end{frame}
