\begin{frame} \frametitle{Симметрия}
	Двое размещают прямоугольники $6 \times 1$ на доске $100 \times 100$. \medskip \pause

\begin{center} \tikz[scale=0.42]{
	\fill[turnb] (-3,-2) rectangle ++(1,6);
	\fill[turna] (3,2) rectangle ++(-1,-6);
	\foreach \x in {-5,...,5} {
		\draw[gray] (-5.5,\x) -- (5.5,\x) (\x,-5.5) -- (\x,5.5);
	}
	\draw[very thick,gray] (0,-5.5) -- (0,5.5) (-5.5,0) -- (5.5,0);
	\draw[very thick, ->] (-1.5,1.5) to[out=20,in=80] (1.5,-1.5);
} \end{center}\medskip

	А если квадраты $2 \times 2$?
\end{frame}

\begin{frame} \frametitle{Тоже симметрия, но в шахматных фигурах}
	Двое по очереди ходят ладьёй по шахматному полю, причём \\
	ходить можно только вниз или влево. \medskip \pause

\begin{center} \tikz[scale=0.42]{
	\draw (0,0) rectangle (8.5,8.5);
	\fill (2.4,2.4) circle[radius=3mm] (6.1,6.1) circle[radius=3mm];
	\draw[very thick,->,turnb] (6.1,5.5) -- (6.1,2.7);
	\draw[very thick,->,turna] (5.8,2.4) -- (3,2.4);
} \end{center}\medskip

	А если доска не квадратная?
\end{frame}

\begin{frame} \frametitle{Камни из кучи}
	Дана куча из 40 камней. За ход можно вынуть из неё от 1 до 7 камней. Проигрывает тот, кто не может сделать ход.\pause

\tikzset{>={Latex[width=0.8mm,length=1.4mm]}}

\begin{center} \tikz[scale=1.35]{
	\foreach \i in {1,...,4} {\draw[gray] (0, 0.5 * \i cm) -- ++(0.6,0);}
	\draw[->] (0.7,2) to[out=320,in=40] (0.7,1.643);
	\draw[->] (0.7,1.643) to[out=335,in=25] (0.7,1.5);
	\draw (0,0) rectangle (0.6,2.5);
} \end{center}

Выигрывает второй, ему надо {\it дополнять ходы противника до 8.}

А если в куче 43 камня?
\end{frame}
