\begin{frame} \frametitle{Симметрия}
	Двое размещают прямоугольники $6 \times 1$ на доске $100 \times 100$. \medskip \pause

\begin{center} \tikz[scale=0.42]{
	\fill[turnb] (-3,-2) rectangle ++(1,6);
	\fill[turna] (3,2) rectangle ++(-1,-6);
	\foreach \x in {-5,...,5} {
		\draw[gray] (-5.5,\x) -- (5.5,\x) (\x,-5.5) -- (\x,5.5);
	}
	\draw[very thick,gray] (0,-5.5) -- (0,5.5) (-5.5,0) -- (5.5,0);
	\draw[very thick, ->] (-1.5,1.5) to[out=20,in=80] (1.5,-1.5);
} \end{center}
\end{frame}

\begin{frame} \frametitle{Тоже симметрия, но в шахматных фигурах}
	Двое по очереди ходят ладьёй по шахматному полю, причём \\
	ходить можно только вниз или влево. \medskip \pause

\begin{center} \tikz[scale=0.42]{
	\draw (0,0) rectangle (8.5,8.5);
	\fill (2.4,2.4) circle[radius=3mm] (6.1,6.1) circle[radius=3mm];
	\draw[very thick,->,turnb] (6.1,5.5) -- (6.1,2.7);
	\draw[very thick,->,turna] (5.8,2.4) -- (3,2.4);
} \end{center}
\end{frame}

\begin{frame} \frametitle{Камни из кучи}
\begin{itemize} \itemsep=4mm
	\item Дана куча из $k$ камней. За ход можно вынуть из неё от 1 до 7 камней. Проигрывает тот, кто не может сделать ход.
	\item Дана куча из $k$ камней. За ход можно вынуть из неё от 1 до 7, \\ а также 9 камней. Проигрывает тот, кто не может сделать ход.
\end{itemize} \bigskip \pause

Выигрывает второй при $k$, делящемся на 8, и первый иначе.
\end{frame}
