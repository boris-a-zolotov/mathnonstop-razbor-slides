\begin{frame} \frametitle{Выигрышные и проигрышные позиции}
Это метод решения задач на игры, который работает почти всегда, \\
если у каждой позиции есть простое описание. \bigskip

Выигрышная позиция — у игрока, начинающего в ней, есть стратегия. \\
Проигрышная — нет стратегии. \bigskip

Например, «последняя» позиция — проигрышная. Позиции, \\
из которых есть прямой ход в «последнюю», — выигрышные.
\end{frame}

\begin{frame} \frametitle{Теорема о характеризации позиций}

\begin{block}{\vspace*{-3ex}}
Выигрышные позиции — такие, из которых есть ход \\
хотя бы в одну проигрышную. \bigskip

Проигрышные позиции — такие, ходы из которых \\
только в выигрышные.\end{block}\bigskip

В соответствии с этим утверждением можно проанализировать \\
все позиции, начиная с конечной. {\it Рассмотрим игры, которые уже были:} \\
{\it две кучи монет, ладья, камни.}
\end{frame}

\begin{frame} \frametitle{Примеры}
В куче $n$ камней, из неё можно вынуть $a_1$, $a_2$, \ldots, $a_k$ камней. \\
0~камней — проигрышная позиция, остальные расставим. \pause

Двое ходят королём по шахматной доске, можно ходить \\
только вниз, влево или вниз-влево. \pause

\begin{center} \tikz[scale=0.46]{
	\fill[turna] (0,0) rectangle (8,8);
	\foreach \x in {0,2,4,6} {
	    \foreach \y in {0,2,4,6} {
		\fill[turnb] (\x,\y) rectangle ++(1,1);
	    }
	}
	\foreach \x in {0,...,8} {
		\draw[gray] (0,\x) -- (8,\x) (\x,0) -- (\x,8);
	}
} \end{center}
\end{frame}

\begin{frame} \frametitle{Игра Ним}
	Имеется $k$ кучек, в них $N_1$, $N_2$, \ldots, $N_k$ камней. Можно вынуть сколько угодно камней, но только из одной кучи. \medskip

\begin{center} \tikz[scale=1.3]{
	\draw (0,0) rectangle ++(0.6,1.2);
	\draw (1,0) rectangle ++(0.6,1.7);
	\draw (2,0) rectangle ++(0.6,0.9);
	\draw (3,0) rectangle ++(0.6,1.5);
	\draw (4,0) rectangle ++(0.6,1.2);
	\draw (5,0) rectangle ++(0.6,0.8);
}\end{center}
\end{frame}


\begin{frame} \frametitle{Ним-сумма}
	Переведём размеры кучек в двоичную систему и сложим \\
	без переноса разрядов.

	То же самое, что разложить в сумму степеней двойки \\
	и посмотреть, каких из них нечётное число. \medskip

\begin{center}\begin{tabular}{cccc}
2 & 8 & 10 & 11 \\
{\small 10} & {\small 1000} & {\small 1010} & {\small 1011}
\end{tabular}

Ним-сумма — $1011$:\\
единиц 1, десятков 3, сотен 0, тысяч 3 \end{center}
\end{frame}

\begin{frame} \frametitle{Свойства ним-суммы} \ \\ [-0.4cm]
	Коммутативна, ассоциативна, $x \oplus x = 0$. \vspace{-0.8cm}
\renewcommand{\sb}{S_{\text{before}}}
\newcommand{\sa}{S_{\text{after}}}

\begin{align*}
&	\sa = \sa \oplus 0 = \sa \oplus \sb \oplus \sb = \\
= &	(x_1 \oplus y_1) \oplus \ldots \oplus (x_k \oplus y_k) \oplus \sb = \\
= & (x_i \oplus y_i) \oplus \sb \\
\end{align*} \vspace{-1.3cm} \pause

$y_i = x_i \oplus (x_i \oplus \sb)$, тогда $x_i \oplus x_i \oplus \sb \oplus \sb = 0$. \pause

\begin{center}\begin{tabular}{cccc}
2 & 8 & 10 & 11 \\
{\small 10} & {\small 1000} & {\small 1010} & {\small 1011} \\
\multicolumn{4}{c}{Ним-сумма — $1011$.} \\ \pause
\phantom{ававав} & \phantom{ававав} & \phantom{ававав} & \phantom{ававав} \\
{\small $1001$} & {\small $11$} & {\small $1$} & {\small $0$} \\
\end{tabular} \end{center}
\end{frame}

\begin{frame} \frametitle{Позиции в игре Ним}
\begin{itemize} \itemsep=5mm
	\item Из любой позиции с ненулевой ним-суммой можно походить \\
	в позицию с нулевой.
	\item При ходе из любой позиции с нулевой ним-суммой \\
	она становится ненулевой.
	\item Значит, позиции с нулевой ним-суммой — проигрышные, \\
	с ненулевой — выигрышные.
\end{itemize}
\end{frame}
