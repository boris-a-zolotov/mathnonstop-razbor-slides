\begin{frame} \begin{center}
	{\Large К чему скриншотить презентацию,\smallskip\\
		когда можно её скачать} \\ [0.9cm]
	{\small Слайды доступны по ссылке: \url{http://bit.ly/spbtym-game-theory}}
\end{center} \end{frame}

\begin{frame} \frametitle{Игры с олимпиад}
	Они же — игры с полной информацией.

\begin{itemize} \itemsep=2.25mm
	\item Множество позиций
	\item Игроки делают ходы по очереди
	\item Игрокам известны все возможные ходы из каждой позиции
	\item На некоторых позициях определяется исход игры, \\
		например — «проигрывает тот, \\
		кто не может сделать ход».
\end{itemize}
\end{frame}

\begin{frame} \frametitle{Кто выигрывает при правильной игре?}
	Правильная игра — никто из игроков не знает, какой ход \\
	его соперник сделает следующим. \bigskip
	
	Нет ни игры «в поддавки», ни игры «в худший случай». \\
	Нельзя сводить рассмотрение такой игры к рассмотрению \\
	одного варианта поведения противника.
\end{frame}

\begin{frame} \frametitle{Что такое выигрышная стратегия}
	Это правило, которое описывает ответы данного игрока \\
	на {\it любые} ходы его противника и при любых \\
	ходах противника приводит к выигрышу. \bigskip
	
	Мы должны уметь отвечать на любой возможный ход — \\
	разумеется, по-разному. Во всех разумных играх стратегия \\
	существует, причём только у одного игрока.
\end{frame}

\begin{frame} \frametitle{Ничья}
	Изредка бывает, что выигрышных стратегий нет, \\
	каждый игрок может не проигрывать. \medskip

\begin{center} \tikz[scale=0.94]{
	\fill[initpos,rotate=90] (1,-0.075) rectangle (1.5,0.075);
	\fill[black,rotate=150] (1,-0.075) rectangle (1.5,0.075);
	\fill[black,rotate=210] (1,-0.075) rectangle (1.5,0.075);
	\fill[black,rotate=270] (1,-0.075) rectangle (1.5,0.075);
	\fill[failpos,rotate=330] (1,-0.075) rectangle (1.5,0.075);
	\fill[black,rotate=30] (1,-0.075) rectangle (1.5,0.075);
} \end{center}
\end{frame}
