\begin{frame} \begin{center}
	{\Large К чему скриншотить презентацию,\smallskip\\
		когда можно её скачать} \\ [0.9cm]
	{\small Слайды доступны по ссылке: \url{http://bit.ly/spbtym-game-theory}}
\end{center} \end{frame}

\begin{frame} \frametitle{Игры с олимпиад}
	Они же — игры с полной информацией.

\begin{itemize} \itemsep=2.25mm
	\item Множество позиций
	\item Игроки делают ходы по очереди
	\item Игрокам известны все возможные ходы из каждой позиции
	\item На некоторых позициях определяется исход игры, \\
		например — «проигрывает тот, кто не может сделать ход».
\end{itemize}
\end{frame}

\begin{frame} \frametitle{Самая базовая игра}
	Есть две кучи по 100 монеток. Можно вынуть сколько угодно \\
	монеток из одной кучи.
	\medskip \pause

\begin{center} \tikz[scale=1.35]{
\draw (-0.5,0) rectangle ++(-0.7,1.2) (0.5,0) rectangle ++(0.7,1.2);
\filldraw[fill=turnb] (-0.5,1.2) rectangle ++(-0.7,0.85);
\filldraw[fill=turna] (0.5,1.2) rectangle ++(0.7,0.85);
} \end{center}\medskip

	{\it Сколько бы первый ни вынул,} второй берёт \\
	столько же из другой кучи.
\end{frame}

\begin{frame} \frametitle{Кто выигрывает при правильной игре?}
	Правильная игра — никто из игроков не знает, какой ход \\
	его соперник сделает следующим. \bigskip
	
	Нет ни игры «в поддавки», ни игры «в худший случай». \\
	Нельзя сводить рассмотрение такой игры к рассмотрению \\
	одного варианта поведения противника.
\end{frame}

\begin{frame} \frametitle{Что такое выигрышная стратегия}
	Это правило, которое описывает ответы данного игрока \\
	на {\it любые} ходы его противника и при {\it любых} \\
	ходах противника приводит к выигрышу. \bigskip
	
	Мы должны уметь отвечать на любой возможный ход — \\
	разумеется, по-разному. Во всех разумных играх стратегия \\
	существует, причём только у одного игрока.
\end{frame}

\begin{frame} \frametitle{Несимметричная позиция}
	Есть две кучи, 100 монеток и 105 монеток. Можно вынуть \\
	сколько угодно монеток из одной кучи.
	\medskip \pause

\begin{center} \tikz[scale=1.35]{
\draw (-0.5,0) rectangle ++(-0.7,1.2) (0.5,0) rectangle ++(0.7,1.2);
\filldraw[fill=turnb] (-0.5,1.2) rectangle ++(-0.7,0.85);
\filldraw[fill=turna] (-0.5,2.05) rectangle ++(-0.7,0.35);
\filldraw[fill=turna] (0.5,1.2) rectangle ++(0.7,0.85);
} \end{center}
\end{frame}