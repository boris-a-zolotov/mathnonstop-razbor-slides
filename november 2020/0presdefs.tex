\usepackage[utf8]{inputenc} \usepackage[T2A]{fontenc} \usepackage[russian]{babel}
\usepackage{amsmath,amssymb,amsthm,mathtools}
\usepackage{graphicx,caption,subcaption}
\usepackage{hyperref}
\usepackage{tikz,xcolor,colortbl,makecell}

\usetikzlibrary{arrows,arrows.meta}

%\theoremstyle{plain}
%\newtheorem{theorem}{Теорема}
%\newtheorem{lemma}[theorem]{Лемма}

%\theoremstyle{definition}
%\newtheorem{definition}{Определение}
%\newtheorem{problem}{Задача}

\usetheme[height=0.97cm]{Rochester}
\usecolortheme{dolphin}

\definecolor{hard}{RGB}{40,40,128}
\definecolor{mnsgold}{RGB}{255,255,130}

% \definecolor{hard}{RGB}{20,100,65}
% \definecolor{mnsgold}{RGB}{240,255,250}

\setbeamercolor{headline}{bg=hard,fg=mnsgold}
\setbeamercolor*{frametitle}{parent=headline}

\setbeamercolor{structure}{fg=hard}
\setbeamercolor{subsection in head/foot}{bg=white,fg=hard}
\setbeamercolor{section in head/foot}{bg=hard,fg=mnsgold}
\setbeamercolor{block title}{bg=hard,fg=mnsgold}

\setbeamertemplate{navigation symbols}{}

\parskip=2.6mm
\def\ll{\left(} \def\rr{\right)}
\def\lag{\left\langle} \def\rag{\right\rangle}

\definecolor{failpos}{RGB}{230,30,20}
\definecolor{initpos}{RGB}{30,20,220}
\definecolor{turna}{RGB}{100,240,110}
\definecolor{turnb}{RGB}{250,140,110}

\newcommand{\vseper}{\vphantom{$\int_{0_0}^{0^0}$}}

\newcommand{\singlepayoff}[2]{\tikz{
	\draw (0,0) node{\vseper #1}; \draw (0.8,0.8) node{\vseper #2};
}}

\newcommand{\rowdescription}[1]{\tikz{
	\draw (0,0) node{\vseper #1}; \draw (0,0.8) node{\vseper};
}}

\newcommand{\coldescription}[1]{\tikz{
	\draw (0,0) node{\vseper}; \draw (0.8,0) node{\vseper #1};
}}

\def\mitem{\medskip\item}
\def\ps{\\ [0.65cm]} \linespread{1.16}
\def\fram#1#2{\begin{frame}\frametitle{#1}#2\end{frame}}
\def\usl#1#2{\begin{block}{#1} #2 \end{block} \medskip\pause}
\def\mov#1#2{\begin{scope}[xshift = #1 cm] #2 \end{scope};}

\def\divsby{\mathrel{\rlap{.}\rlap{\raisebox{0.55ex}{.}}\raisebox{1.1ex}{.}}}
\definecolor{starvert}{RGB}{110,175,230}
\newcommand{\vfi}{\varphi}
\newcommand{\litem}{\vspace{0.5cm}\item}

\newcommand{\nkstar}[2]{
	\foreach \i in {1,...,#1} {\draw[thick] (90 + 360 * \i / #1 : 3.5)
		-- (90 + 360 * \i / #1 + 360 * #2 / #1 : 3.5);}
	\foreach \i in {1,...,#1} {\fill[starvert] (90 + 360 * \i / #1 : 3.5) circle[radius=0.16cm];}
}

\DeclarePairedDelimiter{\lr}{(}{)}
