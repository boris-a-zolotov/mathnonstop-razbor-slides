\documentclass[aspectratio=1610,12pt]{beamer}
\definecolor{spoiler}{gray}{0.55}

\usepackage[utf8x]{inputenc}
\usepackage[russian]{babel}
\usepackage{hyperref}
\usepackage{xcolor,wrapfig,sistyle}
\usepackage{tikz,makecell}

\usetheme{Berlin}
\usecolortheme{dolphin}

\definecolor{hard}{RGB}{30,65,140}
\definecolor{soft}{RGB}{70,100,170}

\setbeamercolor{structure}{fg=hard}
\setbeamercolor{subsection in head/foot}{bg=soft,fg=white}
\setbeamercolor{section in head/foot}{bg=hard,fg=white}
\setbeamercolor{block title}{bg=hard,fg=white}

%%%%%%%%%%%%%%%%
%%%%%%%%%%%%%%%%

\def\fram#1#2{\begin{frame}\frametitle{\bf #1}#2\end{frame}}
\def\scolon{\rlap{,}\raisebox{0.8ex}{,} }
\def\mitem{\medskip\item}

\def\ll{\left(} \def\rr{\right)}
\def\ps{\\ [0.8cm]}

\def\usl#1{\begin{block}{Условие} #1 \end{block} \medskip\pause}
\def\mov#1#2{\begin{scope}[xshift = #1 cm] #2 \end{scope};}

\def\mitem{\medskip\item}

%%%%%%%%%%%%%%%%
%%%%%%%%%%%%%%%%

\title[Математика НОН-СТОП $\mid$ Семинар]
	{\bfseries Принципы составления заданий \\
		и система оценивания \\
		Олимпиады «Математика НОН-СТОП»}

\author[Б.А.\,Золотов, Д.Г.\,Штукенберг]
	{СПбАППО \\ \vspace{0.3cm} Методическая комиссия Олимпиады}

\institute[\textcolor{white}{«Время науки», ЛНМО, СПбАППО}]{}

\date{21 апреля 2018}

%%%%%%%%%%%%%%%%
%%%%%%%%%%%%%%%%

\begin{document}
\section[Про]{Pro—tasks}

\fram{Профильные задачи}{
	Симуляция настоящего научного исследования, доступного для школьника. \ps
	Предлагаем всем ученикам ФМЛ, так как это формат, с которым они ещё не сталкивались и в котором не «вынесут» всех остальных.
}

\fram{Взять известный факт из алгебры...}{
	И попросить школьников доказать его в наглядной и понятной форме. \ps
	Пример — задача про «звёзды»: дан правильный $n$--угольник, соединяем его вершины, находящиеся на равном расстоянии. \medskip
\begin{center}
	\tikz{
		\draw[very thick, rotate=-48, color={rgb:black,1.4;white,8.6}]
			(90:2cm) -- (234:2cm) -- (378:2cm)
			-- (522:2cm) -- (666:2cm) -- cycle;
		\draw[very thick, rotate=-24, color={rgb:black,4;white,6}]
			(90:2cm) -- (234:2cm) -- (378:2cm)
			-- (522:2cm) -- (666:2cm) -- cycle;
		\draw[very thick] (90:2cm) -- (234:2cm) -- (378:2cm)
			-- (522:2cm) -- (666:2cm) -- cycle;
		\draw (3.5 cm, -1.6cm) node {$(15,6)$--звезда};
		\draw (-3.5 cm, -1.6cm) node {\phantom{$(15,6)$--звезда}}
	}
\end{center}}

\def\gcd{\text{НОД}\,} \def\vfi{\varphi}
\fram{$n =\sum\limits_{d \mid n} \vfi(d)$}{
$(n,k)$--звезда состоит из $\gcd(n,k)$ ломаных: в частности, из одной ломаной, когда $n$ и $k$ взаимно просты. \ps
Отсюда есть ровно $\vfi(\tfrac{n}{\ell})$ звёзд, состоящих из $\ell$ ломаных. \ps
$\tfrac{n}{\ell}$ пробегает все делители $n$, а каждая звезда состоит из скольки-то ломаных. Всего звёзд $n$.}








\end{document}