\section[Про]{Pro—tasks}

\fram{Профильные задачи}{
	Симуляция настоящего научного исследования, доступная для школьника. \ps
	Предлагаем всем ученикам ФМЛ, так как это формат, с которым они ещё не сталкивались и в котором не «вынесут» всех остальных.
}

\fram{Взять известный факт из алгебры...}{
	...И попросить школьников доказать его в наглядной и понятной форме. \ps
	Пример — задача про «звёзды» (2017): дан правильный $n$--угольник, соединяем его вершины, находящиеся на равном расстоянии. \medskip
\begin{center}
	\tikz{
		\draw[very thick, rotate=-48, color={rgb:black,1.4;white,8.6}]
			(90:2cm) -- (234:2cm) -- (378:2cm)
			-- (522:2cm) -- (666:2cm) -- cycle;
		\draw[very thick, rotate=-24, color={rgb:black,4;white,6}]
			(90:2cm) -- (234:2cm) -- (378:2cm)
			-- (522:2cm) -- (666:2cm) -- cycle;
		\draw[very thick] (90:2cm) -- (234:2cm) -- (378:2cm)
			-- (522:2cm) -- (666:2cm) -- cycle;
		\foreach \x in {0,...,14} {\draw (90-24*\x:2.25cm)
			node{\scriptsize\x};};
		\draw (4 cm, -1.6cm) node {$(15,6)$--звезда};
		\draw (-4 cm, -1.6cm) node {\phantom{$(15,6)$--звезда}}
	}
\end{center}}

\def\gcd{\text{НОД}\,} \def\vfi{\varphi}
\fram{$n =\sum\limits_{d \mid n} \vfi(d)$}{
$(n,k)$--звезда состоит из $\gcd(n,k)$ ломаных: в частности, из одной ломаной, когда $n$ и $k$ взаимно просты. \ps
Отсюда есть ровно $\vfi(\tfrac{n}{\ell})$ звёзд, состоящих из $\ell$ ломаных. \ps
$\tfrac{n}{\ell}$ пробегает все делители $n$, а каждая звезда состоит из скольки-то ломаных. Всего звёзд $n$.}

\def\D{_{\text{\footnotesize\normalfont Д}}} \def\R{_{\text{\footnotesize\normalfont Р}}}
\fram{Системы счисления}{
Задачи на системы счисления комбинируют чисто технические навыки и неожиданные факты: \ps
\bf{«Десятичная римская система счисления» (2018):}
\begin{align*}
	4047215\R & = 4000000 + 0 - (40000 - (7000 + 200 + 10 - 5)) \\
		& = 3967205\D. \vphantom{\int} \\
	150\D & = 1850\R.
\end{align*}
}

\fram{Неожиданные свойства этой системы счисления}{
\begin{itemize}
	\item Если $S\R = N$, то $0S\R = -N$\scolon
	\item Число может иметь две записи: $1\R = 19\R = 1 \in \mathbb N$\scolon
	\item Число может не иметь записей: 121\scolon
	\item Как изменяются признаки делимости?
\end{itemize}\ \ps
«Задача с открытым концом»: жюри не известен критерий наличия «римской» записи у числа и как установить количество записей. \ps
{\it В двоичной римской с.с. записей не более двух...}}