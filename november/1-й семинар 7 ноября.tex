\documentclass[aspectratio=1610,12pt]{beamer}
\definecolor{spoiler}{gray}{0.55}

\usepackage[utf8x]{inputenc}
\usepackage[russian]{babel}
\usepackage{xcolor,wrapfig,sistyle}
\usepackage{tikz}

\usetheme{Berlin}
\usecolortheme{dolphin}

\definecolor{hard}{RGB}{30,65,140}
\definecolor{soft}{RGB}{70,100,170}

\setbeamercolor{structure}{fg=hard}
\setbeamercolor{subsection in head/foot}{bg=soft,fg=white}
\setbeamercolor{section in head/foot}{bg=hard,fg=white}
\setbeamercolor{block title}{bg=hard,fg=white}

%%%%%%%%%%%%%%%%
%%%%%%%%%%%%%%%%

\def\fram#1#2{\begin{frame}\frametitle{\bf #1}#2\end{frame}}
\def\scolon{\rlap{,}\raisebox{0.8ex}{,} }
\def\mitem{\medskip\item}

\def\ll{\left(} \def\rr{\right)}
\def\ps{\\ [0.8cm]}

\def\usl#1{\begin{block}{Условие} #1 \end{block} \medskip\pause}
\def\mov#1#2{\begin{scope}[xshift = #1 cm] #2 \end{scope};}

%%%%%%%%%%%%%%%%
%%%%%%%%%%%%%%%%

\title[Математика НОН-СТОП $\mid$ Семинар]
	{\bfseries Принципы составления заданий \\
		и система оценивания \\
		Олимпиады «Математика НОН-СТОП»}

\author[Б.А.\,Золотов, Д.Г.\,Штукенберг]
	{СПбАППО \\ \vspace{0.3cm} Методическая комиссия Олимпиады}

\institute[\textcolor{white}{«Время науки», ЛНМО, СПбАППО}]{}

\date{21 апреля 2018}

%%%%%%%%%%%%%%%%
%%%%%%%%%%%%%%%%

\begin{document}
\section[Приветствие]{Hello!}
\begin{frame}\titlepage\end{frame}

\section[Дисклеймер]{Disclaimer}

\fram{Не верьте им!}{
	Основная цель олимпиады — избегать появления «сообщества профессионалов», которые умеют её решать. \ps
	То, что вам сейчас расскажут, — не руководство по подготовке и не классификация задач 2019 года\scolon \ps
	А набор наблюдений, касающихся заданий прошлых лет.
}

\section[0]{Zeroth principle /A bad joke/}
\fram{Главный принцип составления заданий}{
	{\Large\bfseries\itshape Аааа, давай уже что-нибудь напишем, \\ [0.1cm]
		две недели до олимпиады!! \\}
}

\section[1]{First principle /Think world, not math/}
\fram{Задачи «о мире»}{
	Нам не хочется быть только математической олимпиадой: \ps
	Мы любим задачи, которые содержат исследование окружающего мира, \ps
	И зачастую не имеют чёткого ответа.
}

\fram{2017-5-2B («Задача-шутка»)}{
\usl{
	Дана 200-этажная башня. Стул с 30 ножками скидывают с её крыши, и одновременно с этим более лёгкий стул совсем без ножек отправляют катиться вниз по лестнице внутри башни. Может ли безногий стул достигнуть земли быстрее, чем летящий?
}
Странно ожидать от задачи-шутки ответа «нет»: возьмём башню с отвесной лестницей и откачанным воздухом.
}

\fram{2017-4-3A}{
\usl{
	Какие буквы русского алфавита можно перерисовать в другие, добавляя линии?
}
Отсортируем буквы по алфавиту и переберём перерисовывания каждой:
\begin{center}\begin{tabular}{lllllll}
	Б → В & \qquad & Е → В & & О → Ю & & Ш → Щ \\
	Г → Б & & И → Й & \qquad & Р → В & & Ь → Б \\
	Г → В & & К → Ж & & С → О & \qquad & Ь → Ы \\
	Г → Т & & Л → Д & & Ц → Щ & & Ь → Ъ
\end{tabular}\end{center}
Но мы исходили из того, что шрифт типографский.
}

\fram{2016-5-4C}{
\usl{
	Пионер Вася хочет научиться выкладывать цифры наименьшим числом спичек. Помогите ему в этом: найдите наименьшее число $k$ такое, что любая цифра может быть выложена из $k$ спичек.
}
$k=4$:
\begin{center} \tikz{
	\mov{-5}{\draw[very thick] (0.4,-0.4) -- (0.4,0.4) -- (-0.4,0.4) -- (-0.4,-0.4) -- cycle;}
	\mov{-4}{\draw[very thick] (0,-0.4) -- (0,0.4);}
	\mov{-3}{\draw[very thick] (0.4,-0.4) -- (-0.4,-0.4) -- (0.359,-0.147)
		-- (0.4,0.652) -- (-0.4,0.652);}
	\mov{-2}{\draw[very thick] (-0.4,0.4) -- (0.4,0.4) -- (0.4,0) -- (-0.4,0)
		-- (0.4,0) -- (0.4,-0.4) -- (-0.4,-0.4);}
	\mov{-0.6}{\draw[very thick] (0,-0.4) -- (0,0.4) -- (-0.692,0) -- (0.107,0);}
	\mov{0}{\draw (0,0) node {$\ldots$};}
	\mov{1}{\foreach \x in {-45,45} \draw[very thick,rotate=\x] (-0.4,0) -- (0.4,0);
		\foreach \y in {-0.2828,0.2828} \draw[very thick] (-0.4,\y) -- (0.4,\y);}
	\mov{2.2}{\draw[very thick] (-0.4,-0.4) -- (0.4,-0.4) -- (0.4,0.4);
		\draw[very thick] (0.4,0) -- (-0.293,0.4) -- (0.507,0.4);}
}\end{center}}

\fram{2016-5-4C}{
	Почему не обойтись меньшим числом спичек? \ps
	8 должна содержать две петли $\Longrightarrow$ два треугольника, не более двух пар общих сторон $\Longrightarrow$ 4 спички. \ps
\begin{center} \tikz{
	\foreach \x in {-45,45} \draw[very thick,rotate=\x] (-0.4,0) -- (0.4,0);
	\foreach \y in {-0.2828,0.2828} \draw[very thick] (-0.4,\y) -- (0.4,\y);
} \end{center}}

%%%%%%%%%%%%%%%%
%%%%%%%%%%%%%%%%

\section[2]{Second principle /Constructive/}

\fram{Констуктивные задачи}{
	То, что может позволить себе участник \\ с небольшим опытом в математике, — \ps
	Придумать пример, удовлетворяющий заданным свойствам. \ps
	Доказательства же обычно требуют некоторой сноровки.
}

\fram{2016-5-4B}{
\usl{
	Могло ли случиться так, что в петином отряде из 20 пионеров имена у всех начинаются с разных букв?
}
Да, могло: \\ \pause
\begin{quote}
	Аустри, Бримир, Вестри, Гандальв, Двалин, \\
	Ёрд, Ингви, Кили, Лит, Мотсогнир, Нии, \\
	Ори, Регин, Судри, Торин, Фили, \\
	Хефти, Эйкинскьяльди, Яри, \\ Пётр
\end{quote}}

\fram{2018-6-2B}{
\usl{
	Укажите, как разрезать изображённую на рисунке фигуру на 6 равных фигур.
}
\begin{center}\tikz{
	\draw [very thick] (0,0) -- ++(1,0) -- ++(1.5,1.5) -- ++(2,-2)
		-- ++(1,0) -- ++(-1,-1) -- ++(-1,0) -- ++(-0.5,-0.5)
		-- ++(-1,0) -- ++(-2,2);
	\pause
	\draw[thick] (1.5,0.5) -- ++(1,0) -- ++(0.5,-0.5) -- ++(1,0)
		-- ++(-1,0) -- ++(0.5,-0.5) -- ++(-1.5,-1.5);
	\draw[thick] (2.5,0.5) -- ++(-1.5,-1.5) -- ++(1,0) -- ++(0.5,0.5)
		-- ++(1,0) -- ++(1,-1);
	\onslide<2->
}\end{center}}


\fram{2017-8-2C}{
\usl{
	Придумайте (или расскажите, как построить) 95-значное число, в котором нет нулей и которое делится на свою сумму цифр.
}
Придумаем число, делящееся на $144=9\cdot 16$:
$$\underbrace{.\ .\ .\ .\ .\ .\ .}_{\begin{minipage}{2cm}
	\scriptsize разрядов — 91, \\
	$\sum$ цифр — 134
\end{minipage}}3232.$$}






















\section[Конец]{Fin}
\renewcommand{\thefootnote}{/*\!/}
\begin{frame}
	\ \\
	\centerline{\huge Спасибо за внимание!\,\footnote{\ Вы можете задать ещё вопросов}}
\end{frame}
\end{document}
