\documentclass[aspectratio=1610,12pt]{beamer}
\definecolor{spoiler}{gray}{0.55}

\usepackage[utf8x]{inputenc}
\usepackage[russian]{babel}
\usepackage{hyperref}
\usepackage{xcolor,wrapfig,sistyle}
\usepackage{tikz,makecell}

\usetheme{Berlin}
\usecolortheme{dolphin}

\definecolor{hard}{RGB}{30,65,140}
\definecolor{soft}{RGB}{70,100,170}

\setbeamercolor{structure}{fg=hard}
\setbeamercolor{subsection in head/foot}{bg=soft,fg=white}
\setbeamercolor{section in head/foot}{bg=hard,fg=white}
\setbeamercolor{block title}{bg=hard,fg=white}

%%%%%%%%%%%%%%%%
%%%%%%%%%%%%%%%%

\def\fram#1#2{\begin{frame}\frametitle{\bf #1}#2\end{frame}}
\def\scolon{\rlap{,}\raisebox{0.8ex}{,} }
\def\mitem{\medskip\item}

\def\ll{\left(} \def\rr{\right)}
\def\ps{\\ [0.8cm]}

\def\usl#1{\begin{block}{Условие} #1 \end{block} \medskip\pause}
\def\mov#1#2{\begin{scope}[xshift = #1 cm] #2 \end{scope};}

\def\mitem{\medskip\item}

%%%%%%%%%%%%%%%%
%%%%%%%%%%%%%%%%

\title[Математика НОН-СТОП $\mid$ Семинар]
	{\bfseries Принципы составления заданий \\
		и система оценивания \\
		Олимпиады «Математика НОН-СТОП»}

\author[Б.А.\,Золотов, Д.Г.\,Штукенберг]
	{СПбАППО \\ \vspace{0.3cm} Методическая комиссия Олимпиады}

\institute[\textcolor{white}{«Время науки», ЛНМО, СПбАППО}]{}

\date{21 апреля 2018}

%%%%%%%%%%%%%%%%
%%%%%%%%%%%%%%%%

\begin{document}
\section[Приветствие]{Hello!}
\begin{frame}\titlepage\end{frame}

\section[Введение]{Quick Intro}

\fram{История состязаний}{
\begin{itemize}
	\item 2010 — первая олимпиада\scolon
        %\mitem 2011 — 418 участников
	\mitem 2012 — три площадки\scolon
	\mitem 2015 — первый профильный вариант\scolon
	\mitem 2016 — 400 участников пишут базовый вариант, 92 --- профильный\scolon\\
        $\phantom{2016}$ — поддержка Фонда «Время Науки»\scolon
	%\mitem 2017 — задачи для 4 класса\scolon\\
	%$\phantom{2017}$ — автоматическое распределение участников\scolon
	\mitem 2018 — 847 участников пишут базовый вариант, 128 --- профильный\scolon\\
	$\phantom{2018}$ — поддержка Фонда Президентских грантов\scolon\\
        $\phantom{2018}$ — включение в Перечень региональных олимпиад и конкурсов\\
	$\phantom{2018 — }$ интеллектуальной направленности \\
	$\phantom{2018 — }$ Комитета по образованию СПб\scolon\\
	%$\phantom{2018}$ — этот семинар\scolon\\
	$\phantom{2018}$ \textcolor{gray}{— {\itshape выход сборника задач.}}
\end{itemize}}

\fram{Не верьте им!}{
	Основная цель олимпиады — избегать появления «сообщества профессионалов», которые умеют её решать. \ps
	То, что вам сейчас расскажут, — не руководство по подготовке и не классификация задач 2019 года\scolon \ps
	А набор наблюдений, касающихся заданий прошлых лет.
}

\section[1]{First principle /Think world, not math/}
\fram{Задачи «о мире»}{
	Нам не хочется быть только математической олимпиадой: \ps
	Мы любим задачи, которые содержат исследование окружающего мира, \ps
	И зачастую не имеют чёткого ответа.
}

\fram{2017-5-2B («Задача-шутка»)}{
\usl{
	Дана 200-этажная башня. Стул с 30 ножками скидывают с её крыши, и одновременно с этим более лёгкий стул совсем без ножек отправляют катиться вниз по лестнице внутри башни. Может ли безногий стул достигнуть земли быстрее, чем летящий?
}
Странно ожидать от задачи-шутки ответа «нет»: возьмём башню с отвесной лестницей и откачанным воздухом.
}

\fram{2017-4-3A}{
\usl{
	Какие буквы русского алфавита можно перерисовать в другие, добавляя линии?
}
Отсортируем буквы по алфавиту и переберём перерисовывания каждой:
\begin{center}\begin{tabular}{lllllll}
	Б → В & \qquad & Е → В & & О → Ю & & Ш → Щ \\
	Г → Б & & И → Й & \qquad & Р → В & & Ь → Б \\
	Г → В & & К → Ж & & С → О & \qquad & Ь → Ы \\
	Г → Т & & Л → Д & & Ц → Щ & & Ь → Ъ
\end{tabular}\end{center}
Но мы исходили из того, что шрифт типографский.
}

\fram{2016-5-4C}{
\usl{
	Пионер Вася хочет научиться выкладывать цифры наименьшим числом спичек. Помогите ему в этом: найдите наименьшее число $k$ такое, что любая цифра может быть выложена из $k$ спичек.
}
$k=4$:
\begin{center} \tikz{
	\mov{-5}{\draw[very thick] (0.4,-0.4) -- (0.4,0.4) -- (-0.4,0.4) -- (-0.4,-0.4) -- cycle;}
	\mov{-4}{\draw[very thick] (0,-0.4) -- (0,0.4);}
	\mov{-3}{\draw[very thick] (0.4,-0.4) -- (-0.4,-0.4) -- (0.359,-0.147)
		-- (0.4,0.652) -- (-0.4,0.652);}
	\mov{-2}{\draw[very thick] (-0.4,0.4) -- (0.4,0.4) -- (0.4,0) -- (-0.4,0)
		-- (0.4,0) -- (0.4,-0.4) -- (-0.4,-0.4);}
	\mov{-0.6}{\draw[very thick] (0,-0.4) -- (0,0.4) -- (-0.692,0) -- (0.107,0);}
	\mov{0}{\draw (0,0) node {$\ldots$};}
	\mov{1}{\foreach \x in {-45,45} \draw[very thick,rotate=\x] (-0.4,0) -- (0.4,0);
		\foreach \y in {-0.2828,0.2828} \draw[very thick] (-0.4,\y) -- (0.4,\y);}
	\mov{2.2}{\draw[very thick] (-0.4,-0.4) -- (0.4,-0.4) -- (0.4,0.4);
		\draw[very thick] (0.4,0) -- (-0.293,0.4) -- (0.507,0.4);}
}\end{center}}

\fram{2016-5-4C}{
	Почему не обойтись меньшим числом спичек? \ps
	8 должна содержать две петли $\Longrightarrow$ два треугольника, не более двух пар общих сторон $\Longrightarrow$ 4 спички. \ps
\begin{center} \tikz{
	\foreach \x in {-45,45} \draw[very thick,rotate=\x] (-0.4,0) -- (0.4,0);
	\foreach \y in {-0.2828,0.2828} \draw[very thick] (-0.4,\y) -- (0.4,\y);
} \end{center}}

%%%%%%%%%%%%%%%%
%%%%%%%%%%%%%%%%

\section[2]{Second principle /Constructive/}

\fram{Конструктивные задачи}{
	То, что может позволить себе участник \\ с небольшим опытом в математике, — \ps
	Придумать пример, удовлетворяющий заданным свойствам. \ps
	Доказательства же обычно требуют некоторой сноровки.
}

\fram{2016-5-4B}{
\usl{
	Могло ли случиться так, что в петином отряде из 20 пионеров имена у всех начинаются с разных букв?
}
Да, могло: \\ \pause
\begin{quote}
	Аустри, Бримир, Вестри, Гандальв, Двалин, \\
	Ёрд, Ингви, Кили, Лит, Мотсогнир, Нии, \\
	Ори, Регин, Судри, Торин, Фили, \\
	Хефти, Эйкинскьяльди, Яри, \\ Пётр
\end{quote}}

\fram{2018-6-2B}{
\usl{
	Укажите, как разрезать изображённую на рисунке фигуру на 6 равных фигур.
}
\begin{center}\tikz{
	\draw [very thick] (0,0) -- ++(1,0) -- ++(1.5,1.5) -- ++(2,-2)
		-- ++(1,0) -- ++(-1,-1) -- ++(-1,0) -- ++(-0.5,-0.5)
		-- ++(-1,0) -- ++(-2,2);
	\pause
	\draw[thick] (1.5,0.5) -- ++(1,0) -- ++(0.5,-0.5) -- ++(1,0)
		-- ++(-1,0) -- ++(0.5,-0.5) -- ++(-1.5,-1.5);
	\draw[thick] (2.5,0.5) -- ++(-1.5,-1.5) -- ++(1,0) -- ++(0.5,0.5)
		-- ++(1,0) -- ++(1,-1);
	\onslide<2->
}\end{center}}


\fram{2017-8-2C}{
\usl{
	Придумайте (или расскажите, как построить) 95-значное число, в котором нет нулей и которое делится на свою сумму цифр.
}
Придумаем число, делящееся на $144=9\cdot 16$:
$$\underbrace{.\ .\ .\ .\ .\ .\ .}_{\begin{minipage}{2cm}
	\scriptsize разрядов — 91, \\
	$\sum$ цифр — 134
\end{minipage}}3232.$$}


\section[3]{Third principle /Count/}

\fram{Ужасный гадкий аккуратный подсчёт}{
	Один из важных навыков для детей — аккуратность\scolon \ps
	Его мы также с удовольствием проверяем.
}

\fram{2018-4-6A}{
\usl{
Из клетчатой бумаги вырезали прямоугольник размером $4 \times 5$ клеток. Сколько на нём можно найти квадратов? А прямоугольников?
}
Заметим, что левый верхний угол прямоугольника размером $a \times b$ может находиться в $(5-a) \cdot (6-b)$ положениях.}

\fram{2018-4-6A}{\footnotesize
\begin{center}
\begin{tabular}{|c|c|c|c|c|}
\hline\ \ &1&2&3&4\\ \hline
1& $(6−1) (5−1) = \mathbf{20}$ & $(6−1) (5−2) = \mathbf{15}$ & $(6−1) (5−3) = \mathbf{10}$ & $(6−1) (5−4) = \mathbf{5} $ \\
\hline
2& $(6−2) (5−1) = \mathbf{16}$ & $(6−2) (5−2) = \mathbf{12}$ & $(6−2) (5−3) = \mathbf{8}$ & $(6−2) (5−4) = \mathbf{4} $ \\
\hline
3& $(6−3) (5−1) = \mathbf{12}$ & $(6−3) (5−2) = \mathbf{9}$ & $(6−3) (5−3) = \mathbf{6}$ & $(6−3) (5−4) = \mathbf{3} $ \\
\hline
4& $(6−4) (5−1) = \mathbf{8}$ & $(6−4) (5−2) = \mathbf{6}$ & $(6−4) (5−3) = \mathbf{4}$ & $(6−4) (5−4) = \mathbf{2} $ \\
\hline
5& $(6−5) (5−1) = \mathbf{4}$ & $(6−5) (5−2) = \mathbf{3}$ & $(6−5) (5−3) = \mathbf{2}$ & $(6−5) (5−4) = \mathbf{1} $ \\
\hline\hline
\ & \multicolumn{4}{|c|}{$\sum = (1+2+\ldots+5) (1 + \ldots + 4) = 15\cdot 10 = 150$.} \\
\hline
\ & \multicolumn{4}{|c|}{$2 + 6 + 12 + 20 = 40$.} \\ \hline
\end{tabular}
\end{center}
}

\section[4]{Fourth principle /Oh, you've been taught/}

\fram{Опираться на школьную программу}{
	Мы всё-таки апеллируем к знаниям, \\
	получаемым участниками в школе.
}

	\def\kmh#1{\SI{#1}{\text{км}/\text{ч}}}
\fram{2017-4-2B}{
\usl{
	Начинающая П. едет на велосипеде без остановок со скоростью \kmh{15}, а опытный Д.\,Г. — со скоростью \kmh{34}, но остановки на отдых отнимают у него столько же времени, сколько он находится в движении. Кто же быстрее?
}
Средняя скорость П. — \kmh{15}: она едет с этой скоростью постоянно. \\ [0.3cm]
Средняя скорость Д.\,Г. на любом участке движения $\ge$\,\kmh{17} и достигает этого значения только в конце остановок. Он быстрее.}

\fram{2016-6-6B}{
\usl{
	Трое лыжников вышли в лес. Лыжники встали на расстоянии по 100 метров друг от друга. В любой момент времени может двигаться только один лыжник, но при этом лишь по прямой, параллельной отрезку, соединяющему двух оставшихся лыжников. Пару часов покатавшись так по лесу, лыжники замерили расстояние друг между другом — получились цифры в 90, 120 и 150 метров. Докажите, что кто-то из лыжников не выполнял правила в течение поездки.
}
Площадь треугольника, образованного лыжниками, \\
	должна быть неизменна:}

\fram{2016-6-6B}{\vspace{-0.6cm}
\begin{center} \tikz{
	\draw[very thick] (0,1.2) -- (2,4) -- (6,1.2) -- cycle;
	\draw[thick,->] (2,4) -- ++(1.5,0);
	\draw[very thick,dashed] (0,1.2) -- (3.5,4) -- (6,1.2);
	\draw[thick,densely dotted] (2,1.2) -- (2,4);
	\draw[thick,densely dotted] (3.5,1.2) -- (3.5,4);
} \end{center} \vspace{0.15cm}
Треугольник 100, 100, 100 — равносторонний\scolon
$$S = \frac{1}{2} \cdot 100 \cdot 100 \cdot \frac{\sqrt{3}}{2} = 2500 \sqrt{3}.$$ \\ [0.3cm]
Треугольник 90, 120, 150 — прямоугольный (как 3, 4, 5)\scolon $S=90 \cdot 60$.}

\section[5]{Fifth principle /RTFM/}

\fram{Прочтите условие, и будет вам счастье}{
	Самое главное при решении задачи — правильно понять, что написано в её условии. \ps
	Можно пытаться довести это до абсолюта.
}

\fram{2018-5-3C}{
\usl{
	Стул с 720 ножками падает с лестницы. Выяснилось, что при падении он потерял в три раза меньше ножек, чем у него бы осталось, потеряй он в три раза меньше ножек, чем у него осталось сейчас. Так сколько же ножек осталось у стула?
}
$$(720 − x) \pause \cdot 3 = \pause 720 - \pause \tfrac{1}{3} \cdot \pause x$$\pause
\vspace{-0.4cm}$$x = 540$$}


\section[6]{Fifth principle /Games/}

\fram{Задачи про игры}{
	Задачи про игры достаточно популярны в математике. \ps
	Поэтому, чтобы не попасть в общий тренд, нужно делать их либо внутренне примитивными, либо вычурными.
}

\fram{2018-6-3C}{
\usl{
	Двое по очереди вырезают из клетчатого квадрата $4 \times 4$ уголки из трёх клеток, причём первый может вырезать только уголки, ориентированные как буква Г, а второй~— только уголки, ориентированные как буква L. Проигрывает тот, кто не может вырезать очередной уголок. У кого из игроков есть выигрышная стратегия?
}
Побеждает первый, если поставит свою фигуру посередине верхней стороны. \\ [0.3cm]
Доказывать это надо перебором: у второго есть пять вариантов хода.}


\definecolor{hoda}{RGB}{120,205,175}
\definecolor{hodb}{RGB}{190,160,205}

\def\lshape#1#2{
	\filldraw[fill=hoda,draw=hoda] (#1, #2 - 1.4) rectangle (#1 + 0.7, #2);
	\filldraw[fill=hoda,draw=hoda] (#1, #2 - 1.4) rectangle (#1 + 1.4,#2 - 0.7);
	\draw[very thick] (#1,#2) -- ++(0,-1.4) -- ++(1.4,0) --
		++(0,0.7) -- ++(-0.7,0) -- ++(0,0.7) -- ++(-0.7,0);
}

\def\gshape#1#2{
	\filldraw[fill=hodb,draw=hodb] (#1, #2 - 1.4) rectangle (#1 + 0.7, #2);
	\filldraw[fill=hodb,draw=hodb] (#1, #2 - 0.7) rectangle (#1 + 1.4,#2);
	\draw[very thick] (#1,#2) -- ++(0,-1.4) -- ++(0.7,0) --
		++(0,0.7) -- ++(0.7,0) -- ++(0,0.7) -- ++(-1.4,0);
	\draw
}

\fram{2018-6-3C}{
\begin{center}
	\tikz{
		\foreach \x in {-4,0,4}
			\mov{\x}{
			  \draw[thick] (-1.4,-1.4) -- (-1.4,1.4) -- (1.4,1.4) -- (1.4,-1.4) -- cycle;
			  \foreach \t in {-0.7,0,0.7} {
			  	\draw[color=gray] (\t,-1.4) -- (\t,1.4);
			  	\draw[color=gray] (-1.4,\t) -- (1.4,\t);
			  } \gshape{-0.7}{1.4};
			};
		\mov{-4}{\lshape{-1.4}{0.7}; \gshape{0}{0};}
		\mov{0}{\lshape{-1.4}{0}; \gshape{0}{0};}
		\mov{4}{\lshape{-0.7}{0}; \gshape{0}{0.7};}
	} \\
\ \\
	\tikz{
		\foreach \x in {-2,2}
			\mov{\x}{
			  \draw[thick] (-1.4,-1.4) -- (-1.4,1.4) -- (1.4,1.4) -- (1.4,-1.4) -- cycle;
			  \foreach \t in {-0.7,0,0.7} {
			  	\draw[color=gray] (\t,-1.4) -- (\t,1.4);
			  	\draw[color=gray] (-1.4,\t) -- (1.4,\t);
			  } \gshape{-0.7}{1.4};
			};
		\mov{-2}{\lshape{0}{0.7}; \gshape{-1.4}{0};}
		\mov{2}{\lshape{0}{0}; \gshape{-1.4}{0};}
	}
\end{center}}


\fram{2017-8-10B}{
\usl{
	Даны две кучи камней: в одной 23 камня, вторая пока пустая. Также дан мешок с 2017 камнями. Разрешены два типа ходов. Можно брать 1, 2, 3 \\ или 4 камня и перекладывать их из первой кучи во вторую. Также можно перекладывать 1, 2, 3 или 4 камня (если они там есть) из второй кучи в первую — при этом столько же камней, сколько взято, нужно выкинуть из мешка в окно. Играют двое\scolon проигрывает тот, кто выкидывает последний камень из мешка. Кто победит при правильной игре?
}
Стратегия для первого: добиться, чтобы перекладывания 2→1 делал только второй.}

\def\vph{$\vphantom{\int\limits_0^0}$}
\fram{2017-8-10B}{
Первый ход: 3 камня из 1 в 2. Затем \ps
\begin{center}\begin{tabular}{|l|l|}
\hline
\vph Ход второго игрока & Ответный ход \\
\hline \hline
\vph $k$ камней 1→2 & $(5-k)$ камней 1→2 (всегда можем) \\
\hline \hline
\vph {\bfseries $k$ камней 2→1} & $k$ камней 1→2 \\ \hline
\end{tabular}\end{center}
}


\documentclass[aspectratio=1610,12pt]{beamer}
\definecolor{spoiler}{gray}{0.55}

\usepackage[utf8x]{inputenc}
\usepackage[russian]{babel}
\usepackage{hyperref}
\usepackage{xcolor,wrapfig,sistyle}
\usepackage{tikz,makecell}

\usetheme{Berlin}
\usecolortheme{dolphin}

\definecolor{hard}{RGB}{30,65,140}
\definecolor{soft}{RGB}{70,100,170}

\setbeamercolor{structure}{fg=hard}
\setbeamercolor{subsection in head/foot}{bg=soft,fg=white}
\setbeamercolor{section in head/foot}{bg=hard,fg=white}
\setbeamercolor{block title}{bg=hard,fg=white}

%%%%%%%%%%%%%%%%
%%%%%%%%%%%%%%%%

\def\fram#1#2{\begin{frame}\frametitle{\bf #1}#2\end{frame}}
\def\scolon{\rlap{,}\raisebox{0.8ex}{,} }
\def\mitem{\medskip\item}

\def\ll{\left(} \def\rr{\right)}
\def\ps{\\ [0.8cm]}

\def\usl#1{\begin{block}{Условие} #1 \end{block} \medskip\pause}
\def\mov#1#2{\begin{scope}[xshift = #1 cm] #2 \end{scope};}

\def\mitem{\medskip\item}

%%%%%%%%%%%%%%%%
%%%%%%%%%%%%%%%%

\title[Математика НОН-СТОП $\mid$ Семинар]
	{\bfseries Принципы составления заданий \\
		и система оценивания \\
		Олимпиады «Математика НОН-СТОП»}

\author[Б.А.\,Золотов, Д.Г.\,Штукенберг]
	{СПбАППО \\ \vspace{0.3cm} Методическая комиссия Олимпиады}

\institute[\textcolor{white}{«Время науки», ЛНМО, СПбАППО}]{}

\date{21 апреля 2018}

%%%%%%%%%%%%%%%%
%%%%%%%%%%%%%%%%

\begin{document}
\section[Про]{Pro—tasks}

\fram{Профильные задачи}{
	Симуляция настоящего научного исследования, доступного для школьника. \ps
	Предлагаем всем ученикам ФМЛ, так как это формат, с которым они ещё не сталкивались и в котором не «вынесут» всех остальных.
}

\fram{Взять известный факт из алгебры...}{
	И попросить школьников доказать его в наглядной и понятной форме. \ps
	Пример — задача про «звёзды»: дан правильный $n$--угольник, соединяем его вершины, находящиеся на равном расстоянии. \medskip
\begin{center}
	\tikz{
		\draw[very thick, rotate=-48, color={rgb:black,1.4;white,8.6}]
			(90:2cm) -- (234:2cm) -- (378:2cm)
			-- (522:2cm) -- (666:2cm) -- cycle;
		\draw[very thick, rotate=-24, color={rgb:black,4;white,6}]
			(90:2cm) -- (234:2cm) -- (378:2cm)
			-- (522:2cm) -- (666:2cm) -- cycle;
		\draw[very thick] (90:2cm) -- (234:2cm) -- (378:2cm)
			-- (522:2cm) -- (666:2cm) -- cycle;
		\draw (3.5 cm, -1.6cm) node {$(15,6)$--звезда};
		\draw (-3.5 cm, -1.6cm) node {\phantom{$(15,6)$--звезда}}
	}
\end{center}}

\def\gcd{\text{НОД}\,} \def\vfi{\varphi}
\fram{$n =\sum\limits_{d \mid n} \vfi(d)$}{
$(n,k)$--звезда состоит из $\gcd(n,k)$ ломаных: в частности, из одной ломаной, когда $n$ и $k$ взаимно просты. \ps
Отсюда есть ровно $\vfi(\tfrac{n}{\ell})$ звёзд, состоящих из $\ell$ ломаных. \ps
$\tfrac{n}{\ell}$ пробегает все делители $n$, а каждая звезда состоит из скольки-то ломаных. Всего звёзд $n$.}








\end{document}



















\section[Конец]{Fin}

\fram{Где нас найти}{
\begin{itemize}
	\item Условия задач 2016–18:
		\centerline{\url{http://mathnonstop.ru/uchastnikam.html}}
	\mitem Электронная почта:
		\centerline{\href{mailto:boris.a.zolotov@yandex.com}{\tt boris.a.zolotov@yandex.com}}
	\mitem Эта презентация:
		\centerline{\url{http://bit.ly/mns-seminar-1}}
\end{itemize}
}

\renewcommand{\thefootnote}{/*\!/}
\begin{frame}
	\ \\
	\centerline{\huge Спасибо за внимание!\,\footnote{\ Вы можете задать ещё вопросов}}
\end{frame}
\end{document}
