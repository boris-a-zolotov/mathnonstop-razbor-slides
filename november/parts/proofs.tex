\fram{2017-4-1C}{
	\usl{Дана таблица $7 \times 7$. В центры её клеток Кузя вбил гвоздики. Проведите линию через все гвоздики так, чтобы сделать при этом как можно меньшее количество поворотов.}
	Вроде как, 12 поворотов — но {\bf нужно объяснить}, почему нельзя меньше.
}

\fram{2017-4-1C}{
	\def\dwt{\draw[very thick]}
\begin{center} \tikz{
\foreach \x in {-3,...,3} {
	\foreach \y in {-3,...,3} {
		\filldraw (0.6 * \x cm, 0.6 * \y cm + 0.1 cm) arc (90:450:0.1);
	};
	\dwt (-1.8,0.6*\x cm) -- (1.8,0.6*\x cm);
};
\foreach \y in {-3,-1,1} {
	\dwt (-1.8,0.6*\y cm) -- (-1.8,0.6*\y cm +0.6 cm);
	\dwt (1.8,0.6*\y cm +1.2cm) -- (1.8,0.6*\y cm +0.6 cm);
}
} \end{center}
	В каждый ряд, кроме, возможно, двух, мы входим и выходим, поэтому должны сделать два поворота.
	$$2 \cdot (7-2) + 2 = 12.$$
}

\fram{2018-5-2C}{
	\usl{Известно, что в Авиаландии пять городов. Из каждого города летает шесть авиарейсов, внутренних или международных. Докажите, что за границы Авиаландии летает чётное количество авиарейсов.}
	Опять же, {\bf нельзя} ограничиваться исключительно конкретным примером, для которого всё верно.
}

\fram{2018-5-2C}{
	Внутренний рейс имеет 2 «конца» в стране, международный — 1 «конец».
	$$2 \cdot \text{внут.} + \text{межд.} = 5 \cdot 6 = 30.$$
	Только отсюда международных чётное количество. 
}

\fram{2018-5-5C}{
	\usl{$12 \oplus 34 = 1234$. \\ Бывает ли так, что $P + Q > P \oplus Q$?}
	У того, что так не бывает, есть вполне {\bf чёткое доказательство}: \pause
	$$P \oplus Q\ =\ 10^k \cdot P + Q\ >\ P + Q.$$
}

\fram{2018-8-11A}{
	\usl{18 крабов и 17 пауков встали в хоровод, имеющий форму восьмёрки. Это значит, что существо, стоящее в центре этой восьмёрки, держит за лапы четверых своих соседей. Известно, что каждый краб держится за лапы исключительно с пауками. Кто стоит в центре восьмёрки --- краб или паук?}
	Участники приводили только пример одной подходящей восьмёрки. Но вдруг есть другие, где в центре стоит другое существо? Их отсутствие {\bf надо доказать}.
}

\fram{2018-8-11A}{
	Сделаем из восьмёрки круглый хоровод, где либо 19\,к — 17\,п, либо 18\,к — 18\,п (в зависимости от того, кто в центре). \ps
	Крабов должно быть {\it не больше}, чем пауков: $2\text{к} \le 2\text{п}$. \ps
	Значит, в центре стоял паук.
}