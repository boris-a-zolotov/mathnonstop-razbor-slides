\fram{2016-6-1B}{
	\usl{В кафе стоит $n$ четырёхногих стульев. Ночью в кафе заходит мальчик Вася и начинает вслепую подпиливать стульям ножки. С утра стул упадёт под посетителем, если у него останутся неподпиленными меньше трёх ножек. Сколько ножек нужно подпилить Васе, чтобы с утра как минимум $m$ посетителей кафе гарантированно упали?}
	Что же такое {\it гарантированно}?
}

\fram{2016-6-1B}{
	Как бы неудачливо Вася ни подпиливал стульям ножки, в любом случае с утра $m$ должны упасть. Ответ — не $2m$! \\ [0.6cm]
\begin{enumerate}[label=–]
	\item Можно подпилить по одной ножке у всех стульев, и никто не упадёт\scolon
	\mitem Можно подпилить больше двух ножек (и вообще все) у всех стульев {\bfseries (кроме одного)}, которые должны упасть.
\end{enumerate}
	\ \\ [0.4cm]
	Получается
	$$n + 3(m-1) + 1\ =\ n+3m-2.$$
}

\fram{2018-4-3B}{
\usl{
	Укажите, как разрезать изображённую на рисунке фигуру на 6 равных фигур.
}
\begin{center}\tikz{
\begin{scope}[scale=0.7]
	\draw [very thick] (0,0) -- ++(1,0) -- ++(1.5,1.5) -- ++(2,-2)
		-- ++(1,0) -- ++(-1,-1) -- ++(-1,0) -- ++(-0.5,-0.5)
		-- ++(-1,0) -- ++(-2,2);
	\pause
	\draw[thick] (1.5,0.5) -- ++(1,0) -- ++(0.5,-0.5) -- ++(1,0)
		-- ++(-1,0) -- ++(0.5,-0.5) -- ++(-1.5,-1.5);
	\draw[thick] (2.5,0.5) -- ++(-1.5,-1.5) -- ++(1,0) -- ++(0.5,0.5)
		-- ++(1,0) -- ++(1,-1);
	\onslide<2->
	\end{scope}
}\end{center}
	Равные фигуры — это {\bf не} равные по площади фигуры.
}

\def\sqhere#1#2{\filldraw[fill=forfill,draw=black,thick]
	(0.5 * #1 cm,0.5 * #2 cm)
	-- (0.5 * #1 cm + 0.5 cm,0.5 * #2 cm)
	-- (0.5 * #1 cm + 0.5 cm,0.5 * #2 cm + 0.5 cm)
	-- (0.5 * #1 cm,0.5 * #2 cm + 0.5 cm)
	-- cycle;}

\fram{2018-5-4A}{
	\usl{Двое по очереди вырезают из клетчатого прямоугольника $5 \times 2018$ фигуру, изображённую на рисунке. Проигрывает тот, кто не может сделать ход. У кого из игроков есть выигрышная стратегия?
\begin{center} \tikz{
	\sqhere{0}{1}
	\sqhere{1}{1}
	\sqhere{1}{2}
	\sqhere{1}{3}
	\sqhere{0}{3}
	\sqhere{0}{4}
	\sqhere{0}{5}
	\sqhere{1}{5}
} \end{center}
	}
	{\bfseries Нельзя} просто поделить площадь на площадь — $(5 \cdot 2018) / 8$.
}

\fram{2018-5-4A}{
	\begin{center} \tikz{
		\sqhere{0}{0}
		\sqhere{1}{0}
		\sqhere{1}{1}
		\sqhere{1}{2}
		\sqhere{0}{2}
		\sqhere{0}{3}
		\sqhere{0}{4}
		\sqhere{1}{4}
		\draw[thick] (-1,0) -- (2,0);
		\draw[thick] (-1,2.5) -- (2,2.5);
		\draw (-1.5,0) node{$\ldots$};
		\draw (-1.5,2.5) node{$\ldots$};
		\draw (2.5,0) node{$\ldots$};
		\draw (2.5,2.5) node{$\ldots$};
		\draw[thick] (-2,0) -- (-2.75,0) -- (-2.75,2.5) -- (-2,2.5);
		\draw[thick] (3,0) -- (3.75,0) -- (3.75,2.5) -- (3,2.5);
	} \end{center} \ \\ [0.4cm]
Стратегия для \underline{первого} игрока: вырезать свою фигуру посередине, получатся два одинаковых куска. Повторять ходы второго симметрично в другом куске. Если он смог вырезать, то и мы сможем.
}