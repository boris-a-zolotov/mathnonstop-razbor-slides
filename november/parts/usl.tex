\fram{2018-7-1B}{
	\usl{В понедельник Сергей растворил пачку красителя в десятилитровом ведре воды. Фёдор вылил из ёмкости 4 литра раствора, долил 4 литра воды и тщательно размешал.
	\smallskip \\
	На следующий день Сергей снова растворил пачку красителя в 10 литрах воды. На этот раз Фёдор вылил из ведра 2 литра раствора, долил 2 литра воды, тщательно размешал — и повторил ту же последовательность действий ещё раз. В какой из дней в ведре осталось больше красителя?}
	Ответ — {\bf не} одинаковое количество красителя.
}

\fram{2018-7-1B}{
	В первый день концентрация после действий Фёдора — $0.6$: теперь красителя как в 6 литрах исходного раствора. \ps
	Во второй день — $0.8 \cdot 0.8 = 0.64 > 0.6$. \ps
	Секрет в том, что во второй день на второй итерации выливался {\it менее концентрированный} раствор.
}

\fram{2018-7-6B}{
	{\it Наша любимая ошибка.} \medskip
	\usl{Два кубика размером $5 \times 5 \times 5$ см едут по транспортёру, причём расстояние между ними равняется 10 см. С данного транспортёра они попадают на следующий, в два раза более быстрый, и дальше едут по нему. Каково расстояние между ними теперь?}
	Ответ — {\bf не} 20 сантиметров.
}

\fram{2018-7-6B}{
	\begin{center} \tikz{
		\filldraw[very thick,fill=forfill,draw=black] (0,0) -- (1,0) -- (1,1) -- (0,1) -- cycle;
		\filldraw[very thick,fill=forfill,draw=black] (3,0) -- (4,0) -- (4,1) -- (3,1) -- cycle;
		\draw[thick] (0,1.2) -- (0,1.4)
			-- (1.5,1.4) node[above]{\footnotesize на 2 умножается это}
			-- (3,1.4) -- (3,1.2);
		\draw[thick,red] (1,-0.2) -- (1,-0.4)
			-- (2,-0.4) node[below]{\footnotesize но не это!}
			-- (3,-0.4) -- (3,-0.2);
	} \end{center} \pause \medskip
	$$2 \cdot (10+5) - 5 = 25.$$
}