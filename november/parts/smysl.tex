\fram{2017-4-2A}{
	\usl{Девочка въезжает в горку длиной 400 метров со скоростью 10 километров в час. Как долго она будет это делать?}
	Нужно перевести км/ч → м/с, иначе как делить одно на другое?!
}

\fram{2017-4-2A}{
	$$400\,\text{м}\ \biggl/\biggr.\ 
	\ll 10\,\text{км/ч}\ \Bigl/\Bigr.\ 3.6\,\tfrac{\text{км/ч}}{\text{м/с}}\rr\ =\
	\ll40 \cdot 3.6\rr\,\text{с}\ =\ 144\,\text{с}.$$ \pause \ \\ [0.4cm]
	\centerline{Всегда нужно проверять \underline{размерность} и \underline{порядок ответа}.}
}

\fram{2018-4-1A}{
	\usl{Сколько дат в году могли бы оказаться на экране цифровых часов в качестве времени? {\itshape Например, 19 июня~— 19:06, а 28 ноября времени не соответствует.}}
	Самая «фантастическая» ошибка из всех: 8395, {\bfseries 21516} дат в году\scolon 59 мая!!
}

\fram{2018-4-1A}{
	День: 1–23\scolon\medskip\\
	Месяц: 1–12\scolon\medskip\\
	$$23 \cdot 12 = 276.$$ \ \\ [0.4cm]
	({\bf Нельзя} 28 ноября → 11:28, про это был следующий пункт.)
}

\fram{2018-6-4B}{
	\usl{Единица длины {\bfseries метр} определена как 1/40'000'000 Парижского меридиана. Вместо скорости — {\itshape темп}: сколько минут тратится на 1\,км.
	\smallskip \\
	Самый быстрый темп, которого умеет достигать моделька самолёта — $0.54$ мин/км. За сколько часов такая моделька долетит вдоль Парижского меридиана от Северного полюса до Южного и обратно?}
	В вещах, касающихся Земли (её охват, часовые пояса) детям лучше не врать. А ещё в минуте {\bf не} 100 секунд.
}

\fram{2018-6-4B}{
	$$40\,000\,\text{км} \cdot 0.54\,\text{мин/км}\ =\ 21\,600\,\text{мин}
	\ =\ 360\,\text{ч}.$$
}

\fram{2018-8-10C}{
	\usl{Несколько велосипедистов отправились в поход. За обедом они в сумме съедают 2 килограмма еды плюс 0.1 кг за каждый килограмм еды, который они везли на себе до этого. Например, если у них было 10 килограммов еды на всех, то на ближайшем обеде они съедят $2 + 0.1 \cdot 10 = 3$ килограмма, а на следующем~— $2 + 0.1 \cdot (10-3) = 2.7$ килограммов. В походе планируется 30 обедов (а велосипедисты не завтракают и не ужинают). Сколько еды им нужно взять с собой, чтобы её хватило на весь поход (и в конце похода не осталось ничего лишнего)?}
	Иногда «здравый смысл» будет привирать. :)
}

\fram{2018-8-10C}{
	Всего надо везти
	$$\frac{20}{9} \cdot \frac{\left(\tfrac{10}{9}\right)^{29}-1}{\tfrac{10}{9}-1}
	\,\text{кг еды — это примерно $404.61$.}$$ \ \\ [0.4cm]
	То есть за первым обедом съедят $\sim 60\,\text{кг}$, а за последним — $2.2\,\text{кг}$.
}