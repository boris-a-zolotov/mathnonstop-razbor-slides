\section[Методика]{What to Teach Kids}

\fram{Навыки участника}{
	Здесь будут общие слова о том, чего мы хотим.
	(Наверняка все это и так знают.) \ps
	\begin{enumerate}[label=\arabic*)]
		\item {\bf Можно ли?} \\
			\qquad Да — привести пример\scolon\\
			\qquad Нет — {\bf доказать}, что нельзя.
		\mitem {\bf Всегда ли?} \\
			\qquad Да — {\bf доказать} это\scolon\\
			\qquad Нет — привести контрпример.
	\end{enumerate}
}

\fram{Ещё навыки участника}{
	\begin{enumerate}[label=\arabic*)]
		\setcounter{enumi}{2}
		\item Умение строить отрицания: \\
		\qquad {\bf не} (для всякого…) = существует такой, что ({\bf не} …)\scolon\\
		\qquad {\bf не} (существует такой, что …) = для всякого ({\bf не} …).
		\mitem Что такое доказательство — \\
		\qquad это обоснованное на каждом шаге рассуждение о том, почему \\
		\qquad верно так и никак иначе. Это {\bf не} приведение одного примера, \\
		\qquad для которого выполняется то, что должно быть верно всегда.
		\mitem Получаемый результат $\gg$ изученные алгоритмы и клише (та же за- \\ дача про игру).
	\end{enumerate}
}

\fram{\sout{Кто виноват?} Что делать?}{
	\begin{enumerate}[label=–]
		\item Если начнёте тренировать детей для олимпиад — то быстро вырастете из МНС (оно и хорошо).
		\mitem Если начнёте специально тренироваться для МНС — \sout{отправим вас писать профиль} мы более непредсказуемы, чем вы думаете.
		\mitem Если просто учите класс — смотреть на то, чтобы дети находили результат и грамотно его обосновывали.
		\mitem Мы бы хотели, чтобы участники умели {\it писать} и {\it высказывать сложную мысль}. Пример:
	\end{enumerate}
}

\fram{2011-9-3C}{
\usl{Найдите наименьшее $x$, для которого выполняются равенства:
$x = a+b+c = d+e+f$, где $a$, $b$, $c$, $d$, $e$ и $f$ --- попарно различные натуральные числа.}
$$a+b+c+d+e+f=2x.$$ \\ [0.2cm]
Но сумма шести наименьших чисел ($1+\ldots+6 = 21$) нечётна, поэтому нужно хотя бы 22, $x=11$.
$$1 + 3 +7 = 2 + 4 + 5 = 11.$$
}

\fram{Для подготовки всерьёз нужно знать}{
	\begin{enumerate}[label=–]
		\item Принцип Дирихле
		\item Делимость: действия с остатками, разложение на простые
		\item Инварианты, раскраски
		\item Игры и стратегии
		\item Индукция
		\item Комбинаторика, $C_n^k$, $A_n^k$
	\end{enumerate}
}

\section[Мета]{Plans and Dating}

\fram{Занятия–консультации для будущих участников}{
Планируем: \\ [0.4cm]
	\begin{enumerate}[label=–]
		\item четверг, 6 декабря, 17:30\scolon
		\item суббота, 8 декабря, 17:30\scolon
		\item (???)
	\end{enumerate} \ \\ [0.4cm]
в школе 564, Обводный канал, 143.
}


\fram{Где нас найти}{
\begin{itemize}
	\item Условия задач 2016–18:
		\centerline{\url{http://mathnonstop.ru/uchastnikam.html}}
	\mitem Электронная почта:
		\centerline{\href{mailto:boris.a.zolotov@yandex.com}{\tt boris.a.zolotov@yandex.com}}
	\mitem Эта презентация:
		\centerline{\url{http://bit.ly/mns-seminar-2}}
\end{itemize}
}