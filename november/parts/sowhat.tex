\section[Методика]{What to Teach Kids}

\fram{Навыки участника}{
	\begin{enumerate}
		\item {\bf Можно ли?} \\
			\qquad Да — привести пример\scolon\\
			\qquad Нет — {\bf доказать}, что нельзя.
		\mitem {\bf Всегда ли?} \\
			\qquad Да — {\bf доказать} это\scolon\\
			\qquad Нет — привести контрпример.
	\end{enumerate}}

\fram{Ещё навыки участника}{
	\begin{enumerate}
		\setcounter{enumi}{2}
		\item Умение строить отрицания: \\
		\qquad {\bf не} (для всякого…) = существует такой, что ({\bf не} …)\scolon\\
		\qquad {\bf не} (существует такой, что …) = для всякого ({\bf не} …).
		\mitem Что такое доказательство — \\
		\qquad это обоснованное на каждом шаге рассуждение о том, почему \\
		\qquad верно так и никак иначе. Это {\bf не} приведение одного примера, \\
		\qquad для которого выполняется то, что должно быть верно всегда.
		\mitem Получаемый результат $\gg$ изученные алгоритмы и клише.
	\end{enumerate}
}

\fram{Подготовка к олимпиаде}{
	\begin{enumerate}
		\item Если начнёте тренировать детей для олимпиад — то быстро вырастете из МНС (оно и хорошо).
		\mitem Если начнёте специально тренироваться для МНС — есть профильный вариант; условия меняются год от года.
		\mitem Если просто учите класс — смотреть на то, чтобы дети находили результат и грамотно его обосновывали.
		\mitem Мы бы хотели, чтобы участники умели {\it писать} и {\it высказывать сложную мысль}.
	\end{enumerate}}
