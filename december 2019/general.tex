\begin{frame}\frametitle{История олимпиады}
\begin{itemize}
	\item 2010 — первая олимпиада;
        %\mitem 2011 — 418 участников
%	\mitem 2012 — три площадки;
%	\mitem 2015 — первый профильный вариант;
	\item 2016 — 400 участников пишут базовый вариант, 92 --- профильный;\\
        $\phantom{2016}$ — поддержка Фонда «Время Науки»;
	%\mitem 2017 — задачи для 4 класса;\\
	%$\phantom{2017}$ — автоматическое распределение участников;
	\item 2018 — 847 участников пишут базовый вариант, 128 --- профильный;\\
        $\phantom{2018}$ — включение в Перечень региональных олимпиад и конкурсов\\
	$\phantom{2018 — }$\quad интеллектуальной направленности;\\
	$\phantom{2018}$ — поддержка Фонда Президентских грантов\\
	$\phantom{2018 — }$\quad Комитета по образованию СПб;\\
	%$\phantom{2018}$ — этот семинар;\\
	\item 2019 — выход сборника задач;\\
        $\phantom{2019}$ — площадки в Бердске (Новосибирская обл.) и Гомеле (Беларусь);\\
        $\phantom{2019}$ — число участников приближается к 2000.
\end{itemize}\end{frame}

\begin{frame}\frametitle{Статистика олимпиады}
\begin{itemize}
        \item восемь площадок;\\
	\item три города: Санкт-Петербург, Бердск (Новосибирская обл.), Гомель (Беларусь);\\
        \item две страны;\\
        \item проблемы с часовыми поясами.
\end{itemize}\end{frame}
