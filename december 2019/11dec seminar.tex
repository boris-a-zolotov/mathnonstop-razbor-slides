\documentclass[aspectratio=1610,12pt,notheorems]{beamer}

\usepackage[utf8x]{inputenc} \usepackage[russian]{babel}
\usepackage{amsmath,amssymb,amsthm,mathtools}
\usepackage{graphicx,caption,subcaption}
\usepackage{hyperref,natbib}
\usepackage{tikz,xcolor}
\usepackage{algorithm,algpseudocode}

\usetikzlibrary{arrows,backgrounds,patterns,%
	matrix,shapes,fit,calc,shadows,plotmarks,snakes}

\theoremstyle{plain}
\newtheorem{theorem}{Теорема}
\newtheorem{lemma}[theorem]{Лемма}

\theoremstyle{definition}
\newtheorem{definition}{Определение}
\newtheorem{problem}{Задача}

\usetheme[height=0.97cm]{Rochester}
\usecolortheme{dolphin}

\definecolor{hard}{RGB}{140,50,50}
\definecolor{mnsgold}{RGB}{240,230,200}

\setbeamercolor{headline}{bg=hard,fg=mnsgold}
\setbeamercolor*{frametitle}{parent=headline}

\setbeamercolor{structure}{fg=hard}
\setbeamercolor{subsection in head/foot}{bg=white,fg=hard}
\setbeamercolor{section in head/foot}{bg=hard,fg=mnsgold}
\setbeamercolor{block title}{bg=hard,fg=mnsgold}

\setbeamertemplate{navigation symbols}{}

\def\mitem{\medskip\item}
\def\ps{\\ [0.65cm]} \linespread{1.16}
\def\usl#1#2{\begin{block}{#1} #2 \end{block} \medskip\pause}

\def\ll{\left(} \def\rr{\right)}
\def\lag{\left\langle} \def\rag{\right\rangle}

\title[Mathnonstop 2019: the seminar]
    {\bfseries Математика НОН-СТОП: \\
	Новое в 2019 году}

\author[\ ]
	{Б. А. Золотов,\qquad Д. Г. Штукенберг \\ \vspace{0.3cm}
		{\small Фонд «Время Науки»}}

\institute[\ ]{\ }

\date{8 октября 2019}

%%%%%%%%%%%%%%
%%%%%%%%%%%%%%

\begin{document}

\frame{\titlepage}

\begin{frame}
\frametitle{Конструктивные задачи}
	Мы всё так же горячо любим задачи на приведение примера. Они наглядные и незамысловатые, при этом могут быть крайне разнообразными. \ps
	
	Разберём несколько таких задач — от более простых к более сложным.
\end{frame}

\begin{frame}
\frametitle{Простые, но не простые-простые}

\usl{7 класс, 9A–B}{Докажите, что для любого $n$ существует натуральное число $N\!$, у которого ровно $n$ различных натуральных делителей.} \vspace{4mm}

В пункте {\bf A} было $n=43$. А ответ —\pause $$N = 2^{n-1}.$$
\end{frame}

\begin{frame}
\frametitle{Аксиомы выборов}

\usl{7 класс, 3A}{На предприятии работают 50 человек, и они выбирают себе начальника. Есть две кандидатуры, Ваня и Даня. Про каждого работника известно заранее, кому он отдаёт предпочтение: 20 человек за Даню, 30 человек за Ваню. \smallskip \\
Голосование проходит по двухтуровой системе: люди делятся на 5 групп по 10 человек, в каждой группе выбирается кандидат, наиболее популярный среди членов этой группы, и затем из 5 ответов выбирается имя, названное большее число раз. \smallskip \\
Разделите работников на группы так, чтобы в большинстве групп выбрали Даню и он победил на выборах, несмотря на изначально меньшее число голосующих за него.}
\end{frame}

\begin{frame}
\frametitle{Аксиомы выборов}

\begin{center} \tikz{
\begin{scope}[rotate=90,scale=1.2]
	\filldraw[fill=gray,draw=gray,opacity=0.32] (0,0) rectangle (1,5);
	\foreach \x in {0,2,5} {\draw[thick, color=gray]
		(0.5 * \x cm, 0) -- (0.5 * \x cm, 5);}
	\foreach \x in {0,10} {\draw[thick, color=gray]
		(0, 0.5 * \x cm) -- (2.5, 0.5 * \x cm);}
	\foreach \x in {1,3,4} {\draw[color=gray, opacity=0.38]
		(0.5 * \x cm, 0) -- (0.5 * \x cm, 5);}
	\foreach \x in {1,...,9} {\draw[color=gray, opacity=0.38]
		(0, 0.5 * \x cm) -- (2.5, 0.5 * \x cm);}

	\draw (0.5,-0.25) node[right]{\large За Даню};
	\draw (1.75,-0.25) node[right]{\large За Ваню};

	\pause
	\begin{scope} [xscale=-1, xshift=-2.5cm]
	\draw[very thick] (0.5,0) -- (0.5,5);
	\draw[very thick] (2.5,1.5) -- ++ (-1,0) --
		++ (0,0.5) -- ++ (-0.5,0) -- ++(0,-2);
	\draw[very thick] (2.5,3.5) -- ++ (-1,0) --
		++ (0,-0.5) -- ++ (-0.5,0) -- ++(0,2);
	\draw[very thick] (2.5,3.5) --
		++ (-1,0) -- ++ (0,-0.5) --
		++ (-0.5,0) -- ++ (0,-1) -- ++ (0.5,0) --
		++ (0,-0.5) -- ++(1,0);
	\end{scope}
\end{scope} } \end{center} \end{frame}

\begin{frame}
\frametitle{Лабиринт}

\usl{7 класс, 8C}{
Путник в лабиринте видит ситуацию вокруг. Помимо этого, никакой другой 
информации и памяти у него нет. Существует ли какой-нибудь набор правил, чтобы он, 
имея только эту информацию, мог дойти до финальной клетки в любом лабиринте?} \vspace{4mm}

Заметим, что поведение путника однозначно определено в простых ситуациях:

\begin{center} \tikz[scale=0.8]{\begin{scope}[xshift=-3cm]
	\fill[pattern=north east lines] (0,0) -- (0,1) -- (1,1) -- (1,0) -- (1.7,0) -- (1.7,1.7) -- (-0.7,1.7) -- (-0.7,0) -- cycle;
	\draw (0,0) -- (0,1) -- (1,1) -- (1,0);
	\filldraw (0.5,0.5) circle[radius=1.2mm]; \end{scope}
\begin{scope}[xshift=2cm]
	\fill[pattern=north east lines] (-0.7,-0.2) rectangle (0,1.2) (1,-0.2) rectangle (1.7,1.2);
	\draw (0,-0.2) -- (0,1.2) (1,-0.2) -- (1,1.2);
	\filldraw (0.5,0.5) circle[radius=1.2mm];
\end{scope}} \end{center}
\end{frame}

\begin{frame}
\frametitle{Лабиринт}

Приведём решение без $T$-образных перекрёстков, чтобы о них не думать: \pause

\begin{center} \tikz[scale=1.14]{
	\draw (-3,0) ++(0.15,0.15) rectangle ++(5.7,3.7);
	\foreach \x in {-3,...,2} {
	    \foreach \y in {0,...,3} {
		\draw[pattern=north east lines] (\x, \y) ++(0.15,0.15) --
		    ++(0,0.7) -- ++(0.7,0) -- ++(0,-0.7) -- ++(-0.7,0);
	    };
	};
	\draw (2,3.65) node{\itshape\footnotesize Ф};
	\filldraw (0,1.5) circle[radius=0.56mm];
} \end{center} \end{frame}
\begin{frame}
\frametitle{Чтение и {\it изменение} авторского условия}
	Мы уже давали задачи, значительная часть решения которых заключалась в их вдумчивом прочтении. \ps
	
	Теперь мы пошли дальше и предложили участникам скорректировать наши условия. Для этого по сути нужно решить задачу «задом наперёд».
\end{frame}

\begin{frame}
\frametitle{Велопоход — 2019}

\usl{7 класс, 10B}{
В августе Саар планирует доехать от Бишкека до Астаны. Она проехала уже
1210 километров. Сверившись с картой, она поняла, что ей осталось ехать втрое
больше, чем расстояние, которое проедет машина, в 4 раза более быстрая, чем Саар,
за время от текущего момента до момента, когда Саар останется столько же,
сколько она проехала сейчас. \smallskip \\
Каково расстояние между Бишкеком и Астаной?}

Пусть осталось ехать $t$\,км. До момента, когда останется 1210, $t-1210$\,км.
\vspace{-3mm}

$$t = 3 \cdot 4 \cdot (t - 1210),\qquad 11t = 12 \cdot 1210,\qquad t = 1320.$$ \vspace{-12mm}

$$1320+1210 = 2530.$$
\end{frame}

\begin{frame}
\frametitle{Велопоход — 2019}

\usl{7 класс, 10C}{
Замените числа 1210 и 4 в условии пункта {\bfseries B} на какие-то другие так,
чтобы ответ в задаче составил 1400 километров — настоящее расстояние между
Бишкеком и Астаной.}

$A$ — сколько уже проехали, $c$ — отношение скоростей машины и велосипеда. \vspace{-5mm}

\begin{align*}
& t = 3c \cdot (t-A),\qquad t = \frac{3cA}{3c-1}. \\
& A + t = A + \frac{3cA}{3c-1} = A \cdot \frac{6c-1}{3c-1}\quad = 1400.
\end{align*}

Например, $A=100$, $c = \frac{13}{36}$.
\end{frame}

\begin{frame}
\frametitle{Ещё проще, ещё доступнее}
	«Математика НОН-СТОП» — олимпиада для всех, и каждый найдёт в ней то, что сможет режить. \ps

	Разберём несколько задач, доступных каждому.
\end{frame}

\begin{frame} \begin{center}
	\Huge{Спасибо за внимание!} \ps
	\small{Сейчас даже ещё не полночь перед докладом. Прикольно.}
\end{center} \end{frame}

\end{document}