\begin{frame}
\frametitle{Ещё проще, ещё доступнее}
	«Математика НОН-СТОП» — олимпиада для всех, и любой участник найдёт в ней то, что сможет решить. \ps

	Разберём несколько задач, доступных каждому.
\end{frame}

\begin{frame}
\frametitle{Конференция анонимных геометров}

\usl{5 класс, 1A}{В комнату, имеющую форму правильного 12-угольника, заходят 124 любителя вычислительной геометрии. Как рассадить их вдоль стен этой комнаты так, чтобы у каждой стены сидело ровно по 11 любителей вычислительной геометрии? \smallskip \\	
	Любителей геометрии можно сажать и в углы комнаты — но не более чем по одному геометру на угол.}
	\vspace{4mm}

$12 \cdot 11 - 124 = 8$. Значит, что в какие-то 8 углов из 12 надо будет посадить геометров.
\end{frame}

\begin{frame}
\frametitle{Незакрученный удар}

\usl{4 класс, 2A}{Шарик катается
по прямоугольнику, не замедляясь. Когда он подъезжает к краю
прямоугольника, он отскакивает от него и продолжает движение.
В каком положении окажется шарик, будучи запущенным
из центра прямоугольника на рисунке, после того как он проедет
24 клетки по диагонали?}

\def\ballandthick{
	\filldraw (0,0) circle[radius=1.2mm];
	\draw[very thick,->] (0,0) -- (-0.72,0.72);
	\draw[very thick] (-1,-1.5) rectangle (1,1.5);
}

\begin{center} \tikz[scale=1.04]{
	\draw (-1.2,0) node[left]{\small \begin{minipage}{4.2cm}\ \end{minipage}};
	\foreach \x in {-2,...,2} {\draw[color=gray] (-0.5 * \x cm, -1.5) -- (-0.5 * \x cm, 1.5);}
	\foreach \x in {-3,...,3} {\draw[color=gray] (-1, -0.5 * \x cm) -- (1, -0.5 * \x cm);}
	\draw[color=gray, thick, dotted] (0,0) -- (-1,1) -- (-0.5,1.5) -- (0.3,0.7);
	\ballandthick;
\pause
	\draw[color=gray, thick, dotted] (0.3,0.7) -- (1,0) -- (-0.5,-1.5) -- (-1,-1) -- (1,1)
	    -- (0.5,1.5) -- (-1,0) -- (0.5,-1.5) -- (1,-1) -- (0,0);
	\ballandthick; \draw[thick] (-1,0) -- (1,0);
	\draw (1.35,0) node[right]{\small \begin{minipage}{5.6cm} Раз в 6 клеток пересекает \\
	    горизонтальную среднюю линию \end{minipage}};
} \end{center} \end{frame}

\begin{frame}
\frametitle{Мы едем, едем, едем, едем, едем, едем...}

\usl{5 класс, 6A}{Проездной на месяц позволяет его владельцу ездить на метро неограниченное число раз, 
стоимость проездного фиксирована и одинакова в любом
месяце. Укажите, какова должна быть стоимость проездного, чтобы при двух ежедневных 
поездках он не окупался бы в феврале, но окупался бы в октябре?
Стоимость разовой поездки в метро равна 45 рублям.}

Октябрь длиннее февраля, поэтому может быть совершено больше поездок. Проездной, таким образом, может быть дешевле стоимости 62 поездок, но дороже стоимости 56 поездок. \vspace{-3mm}

$$28 \cdot 45 \cdot 2 = 2520 < S < 2790 = 31 \cdot 45 \cdot 2.$$
\end{frame}