\begin{frame}
\frametitle{Чтение и {\it изменение} авторского условия}
	Мы уже давали задачи, значительная часть решения которых заключалась в их вдумчивом прочтении. \ps
	
	Теперь мы пошли дальше и предложили участникам скорректировать наши условия. Для этого по сути нужно решить задачу «задом наперёд».
\end{frame}

\begin{frame}
\frametitle{Велопоход — 2019}

\usl{7 класс, 10B}{
В августе Саар планирует доехать от Бишкека до Астаны. Она проехала уже
1210 километров. Сверившись с картой, она поняла, что ей осталось ехать втрое
больше, чем расстояние, которое проедет машина, в 4 раза более быстрая, чем Саар,
за время от текущего момента до момента, когда Саар останется столько же,
сколько она проехала сейчас. \smallskip \\
Каково расстояние между Бишкеком и Астаной?}

Пусть осталось ехать $t$\,км. До момента, когда останется 1210, $t-1210$\,км.
\vspace{-3mm}

$$t = 3 \cdot 4 \cdot (t - 1210),\qquad 11t = 12 \cdot 1210,\qquad t = 1320.$$ \vspace{-12mm}

$$1320+1210 = 2530.$$
\end{frame}

\begin{frame}
\frametitle{Велопоход — 2019}

\usl{7 класс, 10C}{
Замените числа 1210 и 4 в условии пункта {\bfseries B} на какие-то другие так,
чтобы ответ в задаче составил 1400 километров — настоящее расстояние между
Бишкеком и Астаной.}

$A$ — сколько уже проехали, $c$ — отношение скоростей машины и велосипеда. \vspace{-5mm}

\begin{align*}
& t = 3c \cdot (t-A),\qquad t = \frac{3cA}{3c-1}. \\
& A + t = A + \frac{3cA}{3c-1} = A \cdot \frac{6c-1}{3c-1}\quad = 1400.
\end{align*}

Например, $A=100$, $c = \frac{13}{36}$.
\end{frame}