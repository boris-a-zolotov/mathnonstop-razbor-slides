\documentclass[a4paper,11pt]{article}

\usepackage{amssymb,amsmath,mathtools,tikz,indentfirst}

\usepackage[russian]{babel}
\usepackage[utf8x]{inputenc}

\usepackage[margin=1.8cm]{geometry}
\pagestyle{empty}

\parskip=1.8mm \parindent=1.2cm

\begin{document}

\section{С новым дрифтом}

Справедливое составление сетки какого-либо чемпионата~— очень сложная задача. Казалось бы, победит всегда сильнейший~— но если слишком слабый участник задержится в чемпионате слишком надолго или, тем более, фаворит вылетит слишком рано, то это не понравится ни публике, ни участникам.

Поэтому абсолютное большинство чемпионатов склоняются либо к сетке «первый с последним», либо к сетке «первый с серединой». Из данной задачи дети могли понять, почему выбирают именно эти сетки. Как показала практика, данная задача выглядит простоватой на фоне остальных в подборке турнира и, возможно, лучше бы смотрелась в регате.

В частности, первый пункт оставляет из всех возможных сеток только две, упомянутые выше. Второй пункт~— о том, что два свойства, кажущиеся естественными, справедливость и непрерывность, выполняются только для сетки «первый с последним».

Третий, четвёртый, шестой пункты~— технические упражнения для участников, чтобы они научились лучше работать с сеткой «первый с последним». Ответ в пятом пункте: сложить бумажку пополам \( j \) раз и проткнуть её иголкой там, где находится номер \( i \)~— игла пройдёт через номера всех возможных соперников.

Идея о складывании пополам естественным образом взывает к понятию кода Грея. Седьмой и восьмой пункты~— как раз о нём. Основное свойство кода Грея доказывается по индукции, а ответ в восьмом пункте~— участник \( i \) может встретиться в \( j \)-ом раунде с теми участниками, код Грея, соответствующий которым, совпадает с \( i \)-ой строчкой кода Грея во всех разрядах, начиная с \( j+1 \)-го.

\section{Каким увидят этот лес внуки, зависит только от тебя}

\newcommand{\deredivline}[5]{
	\draw (#1,#2) --
		++(#5 * #1 cm - #5 * #3 cm, #5 * #2 cm - #5 * #4 cm);
	\draw (#3,#4) --
		++(#5 * #3 cm - #5 * #1 cm, #5 * #4 cm - #5 * #2 cm);
}

Эта задача увлекательна тем, что вдохновлена самым реальным явлением. На месте участников турнира любой бы рванул в лес под Толмачёво~— пытаться подобрать правильный ракурс и увидеть нужную фразу~— но три часа пути от Санкт-Петербурга, кажется, остановили участников: время на дорогу можно потратить на решение задач.

Пункты этой задачи можно довольно точно разделить на две группы: сравнительно техническое применение идей, лежащих на поверхности, и едва ли не открытые вопросы, ответ на которые невозможно получить без самого глубокого исследования.

В данном комментарии мы приведём те самые лежащие на поверхности идеи, без которых невозможно мыслить себе решение данной задачи.

Вначале, разумеется, необходимо понять, как выглядят области, внутри которых перестановка неизменна, каковы их границы. Этими самыми границами являются прямые, проходящие через деревья, но {\itshape исключая} отрезки между деревьями. Пример границ и, соответственно, областей для конфигурации из пяти деревьев приведён на рисунке ниже. Следует, однако, предостеречь читателя: наблюдаемые перестановки могут оказаться одинаковыми в каких-нибудь двух областях, не имеющих общей стороны.

\begin{figure}[ht] \centering \begin{tikzpicture}[scale=0.58]
	\fill (0,0) circle[radius=1.2mm] node[right]{\( 1 \)};
	\fill (4,1) circle[radius=1.2mm] node[left]{\( 2 \)};
	\fill (2,0.25) circle[radius=1.2mm] node[below]{\( 3 \)};
	\fill (2,-1.5) circle[radius=1.2mm] node[above]{\( 4 \)};
	\fill (5.5,-2) circle[radius=1.2mm] node[left]{\( 5 \)};

	\deredivline{0}{0}{4}{1}{1.3}
	\deredivline{0}{0}{2}{0.25}{2.85}
	\deredivline{0}{0}{2}{-1.5}{2.85}
	\deredivline{0}{0}{5.5}{-2}{0.9}
%
	\deredivline{4}{1}{2}{0.25}{2.5}
	\deredivline{4}{1}{2}{-1.5}{1.2}
	\deredivline{4}{1}{5.5}{-2}{1.65}
%
	\deredivline{2}{0.25}{2}{-1.5}{3.6}
	\deredivline{2}{0.25}{5.5}{-2}{1.6}
%
	\deredivline{2}{-1.5}{5.5}{-2}{2.1}
\end{tikzpicture} \end{figure}

А вот решения пунктов 7 и 8 восходят к задаче из пятого класса: \( \ell \) прямых, проведённых на плоскости, делят плоскость на \( 1 + \frac{\ell (\ell + 1)}{2} \) областей. В случае данной задачи количество прямых \( \ell = \binom{n}{2} \), так как прямые соответствуют парам деревьев.

Чтобы решить пункт 7, надо вычесть из количества областей \( \binom{n}{2} \), так как отрезки между деревьями исключены из прямых и, соответственно, делить области не могут. В пункте 8 надо также заметить, что прямые в нашей задаче лежат не в общем положении: много прямых проходят через одну точку~— дерево. Соответственно, при добавлении всех прямых, проходящих через одно дерево, количество областей, которые они добавят к разбиению, будет постоянным.

\section{Велики велики}

Суть задачи в том, что стандартное линейное отношение порядка на натуральных числах заменили другим отношением. Теперь некоторые числа несравнимы друг с другом, но при этом, если мы хотим посчитать количество «шагов»~— длину возрастающей цепочки в данном диапазоне~— оно будет меньше, чем при стандартном отношении порядка.

В пунктах 1–4 требуется увеличивать либо все разряды числа сразу, либо разряды по очереди. Соответственно, количество шагов внутри чисел данной разрядности равно либо минимальной разности между 9 и цифрой в числе, либо сумме таких разностей.

Если наименьшая и наибольшая цены имеют разное количество разрядов, то нужно сложить длины возрастающих последовательностей внутри всех промежуточных разрядностей.

В пунктах 5–6 за увеличение числа в смысле введённого нами отношения отвечают уже не все разряды. Участникам надо было попробовать построить возрастающие последовательности и убедиться в том, что, кажется, достаточно следить только за поведением первой половины разрядов числа.

\section{Растут вниз (счетовод Щоща)}

Основная часть решения этой задачи, в которой участник понимает, как строить примеры в данном сеттинге,~— пункты 2 и 3. На них мы и заострим своё внимание.

Существование дерева произвольной глубины, «ответ» в котором зависит от одного листа, доказывается по индукции. Для глубины 1 такое дерево имеет листья 1, 0 и наш неопределённый лист.

Переход доказывается следующим образом: построим дерево глубины \( k+1 \), у которого два поддерева в корне дают определённый ответ, 1 или 0, а оставшееся поддерево даёт неопределённый ответ.

Для этого возьмём пример дерева глубины \( k \), который существует по предположению индукции, и сделаем поддеревьями в корне три его копии: в одной лист под вопросом заменён на 1, в другом на 0 и в третьем~— является листом под вопросом.

\section{Средненько}

Пункт 1 очевиден, достаточно расписать формулы средних арифметических по определению. В пунктах 2–3 рассмотрим набор следующего вида: \( k-1 \) раз взят ноль, и \( N \) раз взято какое-то большое число \( \alpha \). Разобьём набор на поднаборы следующим образом: каждый ноль в свой отдельный поднабор, и все \( \alpha \) в один общий поднабор. Тогда отношение средних арифметических~— взятого стандартным образом и по поднаборам~— будет равно \[ \frac{N}{N+k-1} \cdot \frac{k}{1} = \frac{Nk}{N+k-1}. \]

Разумеется, правильно выбирая \( N \) и \( k \), такое отношение можно сделать сколь угодно большим.

В пункте 4 надо отсортировать длины наборов и отсортировать \( n \) данных чисел набора \( M \). Чтобы увеличить среднее арифметическое, надо помещать самые большие числа в самый маленький набор, а чтобы уменьшить среднее арифметическое~— самые большие числа в самый большой набор. В пункте 5 число \( a_k \) нужно включить в набор в количестве \( 2^{k-1} \) копий.

\section{Растут вниз (игра)}

Существуют различные техники общего анализа игр, когда игры, совершенно, на первый взгляд, разные по постановке, сводятся к решению более или менее фиксированной задачи. В двух задачах регаты мы хотели продемонстрировать участникам одну из таких техник: сведение любой игры к спуску по какому-то дереву.

Дело в том, что спуск по дереву как таковой исследовать совсем несложно. Будем отмечать вершины дерева, в которых есть выигрышная стратегия для начинающего там игрока, единицей и вершины без выигрышной стратегии нулём. Тогда, как говорит нам пункт 3 первой задачи, если известны все числа в потомках данной вершины, число в данной вершине~— единица минус их произведение. Таким образом можно определить, выигрывает ли игрок, начинающий в корне дерева.

Мы явно попросили участников свести к спуску по дереву две игры, связанные с хождением по графам. Если игроки начинают в фиксированной вершине произвольного графа~— не дерева, то дерево по нему строится так: к каждой вершине дерева «приписана» вершина графа, и дети вершины дерева~— те соседи приписанной к нему вершины графа, которые не были приписаны ни к одному предку данной вершины дерева. К корню, разумеется, приписана вершина, в которой начинается игра.

Если же начальная вершина не фиксирована и выбирается первым игроком, то такая игра сводится к игре с фиксированной начальной вершиной. Создадим столько копий графа, сколько в нём есть вершин. Возьмём отдельно начальную вершину~— из неё будет ходить первый игрок.

Каждую копию графа подвесим за одну из её вершин к начальной вершине. Получится граф, в котором первый игрок своим ходом переходит в одну из копий исходного графа, и игра продолжается в этой копии. То есть, мы можем избавиться от произвольной начальной вершины, заменив её фиксированной начальной вершиной в несколько большем графе. А хождение по произвольному графу сводится к спуску по дереву так, как описано выше.

Вообще, любая игра может быть сведена к спуску по дереву. Чтобы осуществить сведение, рассмотрим все позиции этой игры. Вершинами дерева будут позиции, и детьми каждой позиции будут те позиции, в которые из данной можно попасть за один ход игры. Позиции, в которых заканчивается игра, как раз окажутся в листьях дерева.

\end{document}
