\documentclass[a4paper,11pt]{article}

\usepackage{amssymb,amsmath}

\usepackage[russian]{babel}
\usepackage[utf8x]{inputenc}

\usepackage[margin=1.8cm]{geometry}
\pagestyle{empty}

\parskip=1.8mm \parindent=1.2cm

\begin{document}

Рассмотрим задачу: найти последнюю цифру числа $2^{124}$. 2 4 8 6 2 4 8 6 — получается цикл длины 4, последняя цифра — 6. Цель нашей лекции — разобраться, почему появляется этот цикл, и научиться устанавливать его длину в общем случае.

Сравнимость по модулю. Развиваем теорию, чтобы решать задачу. Число сравнимо со своей последней цифрой mod 10.

Остаток — число, сравнимое с данным, выбранное из данного диапазона $0 \ldots n-1$. Для других применений диапазон используется другой, например $-\frac{n}{2} \ldots \frac{n}{2}$.

Когда нужно решить какую-то задачу про сравнимость по модулю, надо подходить к её решению сухо и формалистично. Свойства сравнимости про сумму, разность и умножение.

Будем прибавлять число к себе. Точно получится ноль, точно получится само это число, можно точно установить длину.

Умножение. Может быть предцикл.

Разобраться со случаем взаимно простого: доказать теорему Эйлера.

Однако цикл может быть меньшей длины: квадратичный вычет.

Когда не взаимно простое: сначала предцикл, потом цикл: принцип Дирихле.






\end{document}