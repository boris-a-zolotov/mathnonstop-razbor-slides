\documentclass[a4paper,11pt]{article}

\usepackage{amssymb,amsmath,mathtools,tikz}

\usepackage[russian]{babel}
\usepackage[utf8x]{inputenc}

\usepackage[margin=1.8cm]{geometry}
\pagestyle{empty}

\parskip=1.8mm \parindent=1.2cm

%%%%%%%%

\DeclarePairedDelimiter{\mset}{ \{ }{ \} }
\definecolor{bluebord}{HTML}{070e47}
\newcommand{\smco}[1]{\multicolumn{3}{c|}{#1}}
\newcommand{\ismco}[1]{\multicolumn{3}{|c|}{#1}}

\begin{document}

\section{Экспорт таблиц проверки, импорт оценок}

\begin{enumerate}

\item Дана последовательность из двенадцати чисел:
	\[ 10\ \ \ 42\ \ \ 23\ \ \ 69\ \ \ 4\ \ \ 
	14\ \ \ 72\ \ \ 21\ \ \ 95\ \ \ 108\ \ \ 
	7\ \ \ 88. \]

Если $a$ и $b$~— два соседние числа в этой последовательности, между ними можно поставить знак сложения (\( a+b \)) или взять максимум (\( \max\mset{a, b} \)). Вам нужно взять максимум в ровно четырёх парах соседних чисел, и оставшиеся числа~— 4 исходных и 4 получившихся максимума~— сложить. Какой максимальный результат может при этом получиться?

\item Дана последовательность из $n$ чисел, между соседними числами можно ставить знак «\(+\)» или брать максимум. Требуется взять максимумы в $k$ парах соседних чисел, \(k \le \tfrac{n}{2}\), и сложить \(n-k\) оставшихся чисел: \(n-2k\) исходных чисел и $k$ максимумов. Опишите, как получить наибольший результат. Что делать в случае, если несколько наименьших чисел стоят в последовательности подряд?

\item Даны три набора неотрицательных чисел:
	\( a_1, \ldots, a_k \), \( b_1, \ldots, b_k \) и \( c_1, \ldots, c_k \).
	Рассмотрим два выражения:
\[ \max\mset{a_1 + \ldots + a_k,\ b_1 + \ldots + b_k,\ c_1 + \ldots + c_k} 
	\text{\ \ и\ \ }
	\max\mset{a_1,b_1,c_1} + \ldots + \max\mset{a_k,b_k,c_k}. \]

	Какое из них больше? В зависимости от выбора чисел $a_i$, $b_i$, $c_i$, какое максимальное отношение значений этих выражений может быть достигнуто?

\item Дано девять неотрицательных чисел: $a_1$, $a_2$, $a_3$, $b_1$, $b_2$, $b_3$, $c_1$, $c_2$, $c_3$. Каково максимальное частное выражений
\begin{multline*}
	\max\mset{a_1,a_2,a_3} + \max\mset{b_1,b_2,b_3} + \max\mset{c_1,c_2,c_3} \\
	\text{и }
	\max\mset{a_1,b_1,c_1} + \max\mset{a_2,b_2,c_2} + \max\mset{a_3,b_3,c_3}\text{?}
	\hspace{1cm}
\end{multline*}

\item Чтобы проставить оценки участнику олимпиады «Математика НОН-СТОП», проверяющий заполняет следующую табличку:

\begin{center} \begin{tabular}{|c|c|c|c|c|c|c|c|c|c|c|c|c|c|c|c|c|c|}
   \hline
   \ismco{1} & \smco{2} & \smco{3} & \smco{4} & \smco{5} & \smco{6} \\
   \hline
   A & B & C & A & B & C & A & B & C & A & B & C & A & B & C & A & B & C \\
   \hline
    & & & & & & & & & & & & & & & & & \\
   \hline
\end{tabular} \end{center}

Пункт A каждой задачи оценивается не более чем в 3 балла, пункт B — не более чем в 6, пункт C — не более чем в 9 баллов. Если проверяющий не ставит в ячейку никакого числа, она считается за 0 баллов.

Чтобы посчитать результат участника, по каждой задаче берётся максимум из оценок, которые участник получил за её пункты, и полученные шесть максимумов складываются.

Пункт 6C очень сложный, никто из участников даже не брался его решать.

Представим, что проверяющий немного ошибся и ставит каждую оценку на одну клетку правее, чем она должна стоять. Какая наибольшая разность (по модулю) между истинной оценкой участника и тем, что выйдет у проверяющего, может получиться?

\item Тот же вопрос для случая, когда задач в олимпиаде не 6, а $n$.

\item Представим теперь, что оценки участника ставятся не в строчку, а в квадрат \( 12 \times 12 \). Баллы могут быть любыми числами от 0 до 10. Пункты задач теперь не три подряд идущих клетки строки, а квадрат \( 3 \times 3 \) внутри большого квадрата. Соответственно, задач 16.

\begin{center} \begin{tikzpicture}[scale=0.28]
   \foreach \i in {0,...,12} {
	\draw[bluebord] (0,\i) -- (12,\i) (\i,0) -- (\i,12);
   }
   \foreach \i in {0,3,6,9,12} {
	\draw[very thick] (0,\i) -- (12,\i) (\i,0) -- (\i,12);
   }
\end{tikzpicture} \end{center}

Проверяющий опять ошибся и ставит каждую оценку на одну клетку правее и ниже, чем она должна стоять. Какая наибольшая разность (по модулю) между истинной оценкой участника и тем, что выйдет у проверяющего, может получиться?

\end{enumerate}

\end{document}