\documentclass[17pt]{extarticle}

\usepackage{extsizes}

\usepackage[paperheight=9.6cm,paperwidth=17cm,margin=0.7cm]{geometry}
\parindent=0mm \parskip=3.2mm

\usepackage{amsmath,amssymb}
\usepackage{graphicx,xcolor,tikz}

\usepackage[utf8x]{inputenc}
\usepackage[russian]{babel}

\pagecolor{black}
\color{white}

\usepackage{mathspec}

\setmainfont[
	Path = f/,
	BoldFont=rb.ttf,
	ItalicFont=ri.ttf,
	BoldItalicFont=rbi.ttf
		]{r.ttf}
\setsansfont[
	Path = f/,
	BoldFont=rbb.ttf,
	ItalicFont=rli.ttf,
	BoldItalicFont=rbbi.ttf
		]{rl.ttf}
		
\setmathfont(Digits)[Path = f/]{rm.ttf}
\setmathfont(Latin)[Path = f/]{rmi.ttf}
\setmathfont(Greek)[Path = f/, Uppercase]{rm.ttf}
\setmathfont(Greek)[Path = f/, Lowercase]{rmi.ttf}

\setmonofont[Path = f/]{rm.ttf}

\def\divsby{\mathop{\rlap{.}\rlap{\raisebox{0.55ex}{.}}\raisebox{1.1ex}{.}}}
\renewcommand{\alpha}{α}
\newcommand{\tr}[1]{\textcolor{red}{#1}}

\begin{document}

\ \\ [1cm]

\begin{center} \Large Циклы остатков \end{center}

\newpage

Рассмотрим задачу: найти последнюю цифру числа
	\[ 2^{124} \]

Посмотрим на первые степени двойки.

\begin{center}
	\tr{2}\quad\tr{4}\quad\tr{8}\quad1\tr{6}\quad3\tr{2}\quad6\tr{4}\quad12\tr{8}\quad25\tr{6}
\end{center}

\newpage

Каждая четвёртая степень двойки \\ заканчивается на 6.

В свою очередь, $124 \divsby 4$, поэтому \\ $2^{124}$ заканчивается на 6. \bigskip

Цель нашей лекции — понять, \medskip \\
1) Почему остатки «зацикливаются» \\ и всегда ли так происходит; \medskip \\
2) Какую длину имеет цикл.

\end{document}
