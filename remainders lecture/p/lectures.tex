\newslide{Математические лекции}

Мы прочли ряд лекций \\
для привлечения детей к математике \\
и в качестве помощи \\
при решении олимпиад

\newslide{Теория игр}

\begin{itemize}
	\item Изолированная тема, не требующая\\ предварительных знаний,
	\item Несомненно полезна\\  на множестве олимпиад,
	\item Наглядна, множество задач,\\ можно вызывать детей к доске.
\end{itemize}

\newslide{Теория игр}

\begin{itemize}
	\item Главный результат: теорема о\\ выигрышных и проигрышных позициях,
	\item Развитие: некооперативные игры,\\ hawk—dove game, равновесие Нэша.
\end{itemize}

\newslide{Циклы остатков}

\begin{itemize}
	\item Задача: найти последнюю цифру числа \( 2^{124} \),
	\item Дети сами замечают зацикливание и умеют \\
	решать такую задачу, но хочется рассказать \\
	им теорию и научить \\
	считать длину цикла.
\end{itemize}

\newslide{Циклы остатков}

	Главный результат: пусть $x$ не делится на $n$, \\
	\( \text{НОД} (x,n) = d \). \\
	Пусть $D$ — делитель $n$, состоящий из простых \( p \mid d \) \\
	в наибольших возможных степенях.\\
	 Пусть $k$~— такое число, что \(x^k\) делится на любое \\
	 простое \( p \mid d \) в степени не меньшей, нежели $D$. Тогда
	\[ x^k \equiv x^{k + \euler\lr*{\frac{n}{D}}} \bmod n. \]

\newslide{Циклы остатков}

\begin{itemize}
	\item Пример: \( x = 2 \cdot 3^2 \cdot 5 = 90 \),\ \ \( n = 2^4 \cdot 3^6 \cdot 7 = 81648 \),
	\item \( k=4 \),\ \ \( \frac{n}{D} = 7 \),\ \ \( 90^{4} \equiv 90^{10} \bmod 81648 \).
	\item Развитие: доказать теорему.
\end{itemize}


\newslide{Функция Эйлера}

\begin{itemize}
	\item Красивый кусок из теории чисел,\\ подводка к циклам остатков,
	\item Начинается с подсчёта \\ на конкретных примерах,
	\item \(\euler\lr*{p}\),\ \ \(\euler\lr*{p^k}\).
\end{itemize}

\newslide{Функция Эйлера}

\begin{itemize}
	\item Главный результат: если $m$ и $n$ взаимно просты, то
		\[\euler\lr*{m \cdot n} = \euler\lr*{m} \cdot \euler\lr*{n}.\]
	\item Развитие: другие функции с таким свойством,\\ научиться считать их {\it свёртки.}
\end{itemize}

\newslide{Разбор задач олимпиады}

\begin{itemize}
	\item Наиболее очевидно, {\it зачем} проводить,
	\item Впечатление, мотивация решать,\\ интерес в процессе\\ очень сильно зависит от подборки задач.
\end{itemize}
