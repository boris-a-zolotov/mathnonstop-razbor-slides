\newslide{Новые приёмы в задачах}

Каждый год мы пишем задачи\\
для олимпиады с нуля, и каждый год находим\\
для себя новые приёмы в задачах,\\
обращаем внимание на что-то новое.

\newslide{4 класс, 4B}

Вдохновлена олимпиадой «только на ответ» ЯКласс.\\ Задача про «пропорцию с хвостом».

Если корабль длиной 250 м идёт на полной скорости,\\ то нужно целиться на 250 м перед носом корабля.\\ Корабль идёт с $\frac{1}{3}$ максимальной скорости.\\ В какую точку нужно целиться?
	\[ \lr*{\lr*{250 + 125} / 3} - 125 = 0. \]

\newslide{5 класс, 6}

\begin{center}
\tikz[scale=0.6]{
	\draw[very thick] (0,0) -- (0,3) -- (1,3); \draw (0.5,-0.7) node {\large $1$};
	\draw[very thick] (2,0) -- (2,3); \draw[very thick] (2,2) -- (3,2); \draw (2.5,-0.7) node {\large $2$};
	\draw[very thick] (4,0) -- (4,3) -- (5,2); \draw (4.5,-0.7) node {\large $3$};
	\draw[very thick] (6,0) -- (6,3); \draw[very thick] (6,2) -- (7,3); \draw (6.5,-0.7) node {\large $4$};
	\draw[very thick] (8,0) -- (8,3) -- (9,3) -- (8,2); \draw (8.5,-0.7) node {\large $5$};
	\draw[very thick] (10,0) -- (10,3);\draw[very thick] (11,2) -- (11,3); \draw (10.5,-0.7) node {\large $6$};
	\draw[very thick] (12,0) -- (12,3) -- (13,3) -- (13,2); \draw (12.5,-0.7) node {\large $7$};
	\draw[very thick] (14,0) -- (14,3); \draw[very thick] (14,2) -- (15,2) -- (15,3); \draw (14.5,-0.7) node {\large $8$};
	\draw[very thick] (16,0) -- (16,3) -- (17,3) -- (17,2) -- (16,2); \draw (16.5,-0.7) node {\large $9$};
} 
\end{center}\vspace{-4mm}

Любое число от 1 до 9999 можно однозначно\\
восстановить по его цистерцианской записи.

\newslide{6 класс, 1A} \vspace{-0.4cm}

Давно интересовавший вопрос:

\begin{center} \includegraphics[width=0.55\paperwidth]{fig/cards-dark} \end{center} \vspace{-0.4cm}

Окажутся ли две карты, изначально бывшие рядом,\\
снова рядом после двух таких перемешиваний?

\newslide{7 класс, 3C}

Задача, которую никто не решит:\\
Придумать число, цифры которого при умножении\\
на \( 1\ldots 6 \) остаются теми же,\\
меняется лишь порядок.

{\small
\begin{center}\begin{tabular}{llll}
   $142857 \cdot 1 = 142857$ & $10^0 \equiv 1$ & $142857 \cdot 4 = 571428$ & $10^4 \equiv 4$ \\
   $142857 \cdot 2 = 285714$ & $10^2 \equiv 2$ & $142857 \cdot 4 = 714285$ & $10^5 \equiv 5$ \\
   $142857 \cdot 3 = 428571$ & $10^1 \equiv 3$ & $142857 \cdot 4 = 857142$ & $10^3 \equiv 6$ \\
\end{tabular}\end{center}}

\newslide{8 класс, 2A}

Аккуратный подсчёт: \begin{center} \tikz[scale=0.9]{
	     \draw[thick] (-2,0) \vnode -- (-1,0) \vnode -- (0,0) \vnode --
	          (0,1) \vnode -- (-1,1) \vnode -- (-2,1) \vnode;
	     \draw[thick] (0,1) -- (1,1) \vnode -- (2,1) \vnode --
	          (3,1) \vnode -- (4,1) \vnode;
	     \draw[thick] (0,0) -- (1,1);
	} \end{center} \vspace{-8mm}

\[C_3^2 + C_3^2 + C_4^2 +
	2 \cdot 3 \cdot 3 + 2 \cdot 3 \cdot 4 + 2 \cdot 3 \cdot 4.\] \vspace{-8mm}

Формулой, как обычно, посчитать быстрее\\ и надёжнее, чем перечисляя все пути.
