\documentclass[a4paper,17pt]{extarticle} \pagestyle{empty}

\usepackage{extsizes}

\usepackage[utf8x]{inputenc}
\usepackage[russian]{babel}

\usepackage[margin=0.5mm]{geometry}

\usepackage{amsmath}
\usepackage{tikz} \usetikzlibrary{}

\begin{document}

\newcommand{\xplotstretch}{0.14} % width of the plot, smaller=better
\newcommand{\yplotstretch}{10} % height of the plot, greater=more jumps
\newcommand{\yplotscale}{1.2} % scale, overall height

\begin{center} \begin{tikzpicture}[declare function={%
	drob(\x) = Mod(\x,1);%
	arctanh(\x) = 0.5*(ln(1+\x)-ln(1-\x));%
	stanh(\x) = \yplotscale * drob(\yplotstretch*tanh(\xplotstretch*\x));%
	invtanh(\x) = arctanh(\x / \yplotstretch) / \xplotstretch;%
	linvtanh(\x) = max(invtanh(\x),-14.77);%
	rinvtanh(\x) = min(invtanh(\x),14.77);}]

\newcommand{\straighttan}

\fill[white] (-10.25,-14.77) rectangle (10.25,14.77); % основной прямоугольник

\begin{scope}[xshift=-8cm,rotate=90]
\foreach \t in {-10,...,9} {
\draw[domain=linvtanh(\t+0.05):rinvtanh(\t+0.95), samples=60, variable=\x, blue, thick]
	plot ({\x},{stanh(\x)});
}
\end{scope}


\end{tikzpicture} \end{center}

\end{document}
