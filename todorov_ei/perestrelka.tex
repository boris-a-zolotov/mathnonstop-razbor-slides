\documentclass[aspectratio=1610,12pt,notheorems]{beamer}

\usepackage[utf8x]{inputenc} \usepackage[russian]{babel}
\usepackage{amsmath,amssymb,amsthm,mathtools}
\usepackage{graphicx,caption,subcaption}
\usepackage{hyperref,natbib}
\usepackage{tikz,xcolor,colortbl,makecell}
\usepackage{algorithm,algpseudocode}
\usepackage{qrcode}

\usetikzlibrary{arrows,backgrounds,patterns,%
	matrix,shapes,fit,calc,shadows,plotmarks,snakes}

\theoremstyle{plain}
\newtheorem{theorem}{Теорема}
\newtheorem{lemma}[theorem]{Лемма}

\theoremstyle{definition}
\newtheorem{definition}{Определение}
\newtheorem{problem}{Задача}

\usetheme[height=0.97cm]{Rochester}
\usecolortheme{dolphin}

\definecolor{hard}{RGB}{145,55,55}
\definecolor{mnsgold}{RGB}{240,235,220}

\setbeamercolor{headline}{bg=hard,fg=mnsgold}
\setbeamercolor*{frametitle}{parent=headline}

\setbeamercolor{structure}{fg=hard}
\setbeamercolor{subsection in head/foot}{bg=white,fg=hard}
\setbeamercolor{section in head/foot}{bg=hard,fg=mnsgold}
\setbeamercolor{block title}{bg=hard,fg=mnsgold}

\setbeamertemplate{navigation symbols}{}

\parskip=2.6mm
\def\ll{\left(} \def\rr{\right)}
\def\lag{\left\langle} \def\rag{\right\rangle}

\definecolor{failpos}{RGB}{230,30,20}
\definecolor{initpos}{RGB}{30,20,220}
\definecolor{turna}{RGB}{100,240,110}
\definecolor{turnb}{RGB}{250,140,110}

\newcommand{\vseper}{\vphantom{$\int_{0_0}^{0^0}$}}

\newcommand{\singlepayoff}[2]{\tikz{
	\draw (0,0) node{\vseper #1}; \draw (0.8,0.8) node{\vseper #2};
}}

\newcommand{\rowdescription}[1]{\tikz{
	\draw (0,0) node{\vseper #1}; \draw (0,0.8) node{\vseper};
}}

\newcommand{\coldescription}[1]{\tikz{
	\draw (0,0) node{\vseper}; \draw (0.8,0) node{\vseper #1};
}}

\newcommand{\myref}[2]{\href{#1}{\texttt{\underline{#2}}}}

\def\mitem{\medskip\item}
\def\ps{\\ [0.65cm]} \linespread{1.16}
\def\fram#1#2{\begin{frame}\frametitle{#1}#2\end{frame}}
\def\usl#1#2{\begin{block}{#1} #2 \end{block} \medskip\pause}
\def\uslnp#1#2{\begin{block}{#1} #2 \end{block} \medskip}
\def\mov#1#2{\begin{scope}[xshift = #1 cm] #2 \end{scope};}

\def\divsby{\mathrel{\rlap{.}\rlap{\raisebox{0.55ex}{.}}\raisebox{1.1ex}{.}}}
\definecolor{starvert}{RGB}{110,175,230}
\newcommand{\vfi}{\varphi}
\newcommand{\litem}{\vspace{0.5cm}\item}

\newcommand{\nkstar}[2]{
	\foreach \i in {1,...,#1} {\draw[thick] (90 + 360 * \i / #1 : 3.5)
		-- (90 + 360 * \i / #1 + 360 * #2 / #1 : 3.5);}
	\foreach \i in {1,...,#1} {\fill[starvert] (90 + 360 * \i / #1 : 3.5) circle[radius=0.16cm];}
}

\newcounter{zadacha}
\setcounter{zadacha}{0}

\newcommand\slprob{
	\frametitle{\thezadacha} \stepcounter{zadacha}
}

\begin{document}

\begin{frame} \frametitle{Правила игры}
\begin{enumerate}
	\item Игра разделены на 14 тактов. Изначально у каждой команды 7 жизней;
	\mitem В конце каждого такта сдаём листочек с ($*$) названием команды\\ ($*$) названием команды, в которую стреляем ($*$) ответом на задачу;
	\mitem Можно не стрелять в другие команды, а провести лечение на +1,\\ но не более семи раз;
	\mitem После 5 правильно сданных задач лечение становится +2, а не +1;
	\mitem Если у команды становится 0 жизней, то первые два хода после этого выстрелы и лечения в 2 раза сильнее. Меньше 0 жизней иметь нельзя.
\end{enumerate}
\end{frame}

\begin{frame} \slprob
	Длину кирпича увеличили на 25\%, а ширину уменьшили на одну треть. На сколько процентов и в какую сторону надо изменить высоту кирпича, чтобы его объём сохранился? 
\end{frame}

\begin{frame} \slprob
	У скольки трёхзначных чисел цифра сотен больше цифры десятков? 
\end{frame}

\begin{frame} \slprob
	Сколько есть способов разместить уголок из трёх клеток внутри прямоугольника размером $5 \times 6$ по линиям клеток? Уголок можно поворачивать.
\end{frame}

\begin{frame} \slprob
	Спортсмен Жан-Грожан ест семь раз в день: нулевой завтрак, первый завтрак, второй завтрак, подобед, надобед, недоужин и переужин. Его нулевой завтрак обыкновенно составляет 1400~ккал, после этого ему остаётся съесть 6600~ккал, чтобы достичь дневной нормы для спортсменов, тем более таких видных и обаятельных.

	Но сегодня годовщина изобретения микроволновки. В честь этого события норма калорий для Жана-Грожана уменьшена вдвое. К сожалению, до Жана-Грожана эту информацию донесли только после нулевого завтрака. Сколько теперь калорий остаётся на остальные шесть приёмов пищи?
\end{frame}

\begin{frame} \slprob
Монахи-цистерцианцы для записи цифр от 1 до 9 использовали следующие символы:\vspace{-3mm}

\begin{center}
\tikz[scale=0.4]{
	\draw[very thick] (0,0) -- (0,3) -- (1,3); \draw (0.5,-0.7) node {\large $1$};
	\draw[very thick] (2,0) -- (2,3); \draw[very thick] (2,2) -- (3,2); \draw (2.5,-0.7) node {\large $2$};
	\draw[very thick] (4,0) -- (4,3) -- (5,2); \draw (4.5,-0.7) node {\large $3$};
	\draw[very thick] (6,0) -- (6,3); \draw[very thick] (6,2) -- (7,3); \draw (6.5,-0.7) node {\large $4$};
	\draw[very thick] (8,0) -- (8,3) -- (9,3) -- (8,2); \draw (8.5,-0.7) node {\large $5$};
	\draw[very thick] (10,0) -- (10,3);\draw[very thick] (11,2) -- (11,3); \draw (10.5,-0.7) node {\large $6$};
	\draw[very thick] (12,0) -- (12,3) -- (13,3) -- (13,2); \draw (12.5,-0.7) node {\large $7$};
	\draw[very thick] (14,0) -- (14,3); \draw[very thick] (14,2) -- (15,2) -- (15,3); \draw (14.5,-0.7) node {\large $8$};
	\draw[very thick] (16,0) -- (16,3) -- (17,3) -- (17,2) -- (16,2); \draw (16.5,-0.7) node {\large $9$};
} 
\end{center}\vspace{-4mm}

Если символ был написан отражённым по вертикали, то его значение умножалось на 10. Если символ был отражён по горизонтали, то его значение умножалось на 100.

Числа от 1 до 9999 записывали единым глифом, совмещая нужные символы вдоль длинной вертикальной черты. Какому числу соответствует символ ниже?\vspace{-3mm}

\begin{center}
\tikz[scale=0.4]{
	\draw[very thick] (0,0) -- (0,3) -- (1,3);\draw[very thick] (-1,1) -- (0,1);\draw[very thick] (-1,2) -- (0,2);
} 
\end{center} \end{frame}

\begin{frame} \slprob
	Художник начал писать картину в последнюю пятницу февраля, а закончил в первую среду марта. Сколько дней работал художник? Укажите все возможные ответы.
\end{frame}

\begin{frame} \slprob
	Когда до полного числа десятков не хватило 2 яиц, их пересчитали дюжинами. Осталось 8 яиц. Сколько было яиц, если их было больше 300, но меньше 400? Найдите все возможные ответы.
\end{frame}

\begin{frame} \slprob
	Баба-Яга сидит перед кучей яиц и нумерует их, начиная с единицы. Как только на очередном яйце записывается номер, кратный четырём, так в куче возникает еще одно новое яйцо. Баба-Яга закончила своё дело, поставив на последнем яйце номер 2006. Сколько яиц было у неё первоначально? 
\end{frame}

\begin{frame} \slprob
Представим себе комнату, имеющую форму многоугольника. Хозяин комнаты миллионер Фридрих фон Нескоро-Богатов хочет повесить красивые картины в некоторых углах комнаты так, чтобы из каждой точки комнаты была видна хотя бы одна картина.

Пусть комната имеет форму, изображённую на рисунке ниже. Каким наименьшим количеством картин сможет обойтись Фридрих? \begin{center} \tikz[scale=0.33,rotate=-43]{
		\draw[thick] (-0.5,-2.5) -- ++(0,5) -- ++(4,4)
			%node[circle,draw=red,fill=red,inner sep=0pt,minimum size=0.3cm]{ }
			-- ++(4,-0.5) -- ++(-4,-0.5) -- ++(-3,-3);
		\draw[thick,rotate=180] (-0.5,-2.5) -- ++(0,5) -- ++(4,4)
		%node[circle,draw=red,fill=red,inner sep=0pt,minimum size=0.3cm]{ }
		-- ++(4,-0.5) -- ++(-4,-0.5) -- ++(-3,-3)
		%node[circle,draw=red,fill=red,inner sep=0pt,minimum size=0.3cm]{ };
	} \end{center}
\end{frame}

\begin{frame} \slprob
Задумано два числа. Если сложить 20\% первого
числа и 30\% второго числа, то получится второе
число. Какое из задуманных чисел больше и во
сколько раз? 
\end{frame}

\begin{frame} \slprob
Найдите восемь последовательных целых чисел, сумма первых пяти из которых равна сумме остальных трёх. 
\end{frame}

\begin{frame} \slprob
Я отпил $\frac16$ чашечки кофе и долил её молоком. Затем выпил $\frac13$
чашечки и долил её молоком. Потом я выпил полчашечки и опять
долил её молоком. Наконец, я выпил полную чашечку. Чего я выпил больше: кофе или молока?
\end{frame}

\begin{frame} \slprob
На уроке физкультуры весь класс выстроился в шеренгу. Сначала
учитель велел рассчитаться на «первый, второй, третий», и каждый третий сделал шаг вперед. По второй команде каждый пятый
из оставшихся сделал шаг назад. После этого в строю осталось 16
человек. Сколько учеников было в классе? Укажите все варианты. 
\end{frame}

\begin{frame} \slprob
В подъезде на первом этаже 2 квартиры, а на всех остальных этажах по 4 квартиры. На каком этаже находится квартира с номером 45? 
\end{frame}

\end{document}

\begin{frame} \frametitle{}
\end{frame}

\begin{frame} \slprob
\end{frame}
