\documentclass[a4paper,11pt,landscape]{article}
\usepackage{tikz,mathspec,makecell,graphicx,qrcode,paracol,titlesec}
\usepackage[russian]{babel}
\usepackage[top=0.7cm,bottom=0.6cm,right=1.8cm,left=1.1cm]{geometry}
\parindent=0.9cm \parskip=1.7mm

\setmainfont[
	Path = f/,
	BoldFont=lb.ttf,
	ItalicFont=li.ttf,
	BoldItalicFont=lbi.ttf
		]{l.ttf}
\setsansfont[
	Path = f/,
	BoldFont=lb.ttf,
	ItalicFont=li.ttf,
	BoldItalicFont=lbi.ttf
		]{l.ttf}

\setmathfont(Digits)[Path = f/]{l.ttf}
\setmathfont(Latin)[Path = f/]{li.ttf}

\titlespacing*{\section}
{0.9cm}{-0.5mm}{-0.5mm}

\begin{document} \thispagestyle{empty}

\begin{center}
    \Large \textbf{Методические рекомендации по построению курса олимпиадной математики}
\end{center}\vspace{-2mm}


\begin{paracol}{2}

\setlength{\columnsep}{0.9cm}

\section{Начальные разделы}

Головоноги. Задачи об улитке. Игры со спичками. Волк, коза и капуста. Козы и колья.

\section{Логика}

Общее утверждение и утверждение о существовании. Построение отрицаний. Доказательство от противного. Логические связки и высказывания. Текстовые задачи. 
\section{Круги Эйлера}

Определение, подсчёт числа предметов с конкретным сочетанием свойств. Мощность множества. Формула включений и исключений.

\section{Делимость}

Чётность. Суммы и произведения Ч и НЧ чисел. Задачи о разложениях. Определение, свойства делимости. Простые числа и их свойства. НОД, НОК. Арифметика остатков. Основная теорема арифметики. Циклы при сложении и возведении в степень. Китайская теорема об остатках. Алгоритм Евклида и его применения. Функция Эйлера, Малая теорема Ферма.

\section{Раскраски и разрезания}

Шахматные доски, вырезание фигур, паркеты. Четыре или несколько цветов. Ходы шахматных фигур и чередования цветов. Задачи на разрезания. 

\section{Комбинаторика}

Сложить или умножить? Простейшие задачи на количество вариантов. Число размещений и сочетаний. Свойства сочетаний, треугольник Паскаля, план города Нью-Йорка. Состояния и формула $a^b$\!. Перестановки с повторениями. Шары и перегородки. Задача о беспорядках. Азы теории вероятности.

\section{Принцип Дирихле}

Принцип Дирихле, обобщенный принцип Дирихле, следствия из принципа Дирихле. Простые и сложные задачи, применение комбинаторики — подсчёт числа возможных случаев.

\section{Инварианты}

Определение: инвариант — то, что не меняется. Чётность, делимость, цвет, площадь, разность и прочие инварианты. Полуинварианты.

\switchcolumn
\setcounter{section}{8}

\section{Графы}

Определения: путь, цикл, висячая вершина, степень, дерево. Сумма степеней вершин и количество рёбер. Дерево — минимальный связный граф. В дереве есть висячая вершина: самый длинный путь. Двудольные графы. Эйлеров и гамильтонов пути и циклы. Формула Эйлера для деревьев и плоских графов.

\section{Игры}

Что такое правильная игра, выигрышная стратегия. Игры-шутки, игры на симметрию, дополнение до $k+1$. Метод проигрышных и выигрышных позиций.

\section{Геометрия на клетчатой бумаге}

Понятие площади и периметра. Элементарные задачи. Формула Пика (с доказательством). Задачи на формулу Пика.

\section{Индукция}

Задача о Ханойской башне, задача о шоколадке $2^n \times 2^n$ без клетки, тождества, геометрические задачи. Доказательство формул для сумм первых $n$ чисел вида $k$, $k^2$\!, $k^3$\!, $2k$, $2k+1$ с помощью индукции. Арифметическая и геометрическая прогрессии. Сильная форма индукции. Числа Фибоначчи и их свойства.

\section{Системы счисления}

Системы счисления с основанием—степенью, смешанные системы счисления: календарь, негапозиционные, уравновешенные, Фибоначчиева, 10-ади-\linebreak
ческая. Факториальная с/с и нумерация перестановок.

\vfill\hrule

\begin{center}
\begin{tabular}{lcclc}
    \makecell[l]{Олимпиада «Математика\\ НОН-СТОП»} &
    \qrcode[hyperlink,height=1.45cm]{https://mathnonstop.ru/} &
    \quad &
    \makecell[l]{Петербургский Турнир \\ юных математиков} &
    \qrcode[hyperlink,height=1.45cm]{https://forms.gle/WuuC2qttGcfETTUm8}
\end{tabular}\vspace{2.5mm}

    \includegraphics[width=8.6cm]{funds}
\end{center}

\end{paracol}

\end{document}
